\documentclass[letter]{article}
\usepackage{amsmath}
\usepackage{amsfonts}
\usepackage{amssymb}
\usepackage{ifthen}
\usepackage{fancyhdr}
\usepackage[usenames,dvipsnames,svgnames,table]{xcolor}
\usepackage{tikz}
\usepackage{enumerate}

%%%
% Set up the margins to use a fairly large area of the page
%%%
\oddsidemargin=.4in
\evensidemargin=.4in
\textwidth=6in
\topmargin=0in
\textheight=9.0in
\parskip=.07in
\parindent=0in
\pagestyle{fancy}

\expandafter\def\expandafter\quote\expandafter{\quote\sf\color{DarkGreen}}

%%%
% Set up the header
%%%
\newcommand{\setheader}[6]{
	\lhead{{\sc #1}\\{\sc #2} %({\small \it \today})
	}
	\rhead{
		{\bf #3} 
		\ifthenelse{\equal{#4}{}}{}{(#4)}\\
		{\bf #5} 
		\ifthenelse{\equal{#6}{}}{}{(#6)}%
	}
}

%%%
% Set up some shortcut commands
%%%
\newcommand{\R}{\mathbb{R}}
\newcommand{\N}{\mathbb{N}}
\newcommand{\Z}{\mathbb{Z}}
\newcommand{\proj}{\mathrm{proj}}
\newcommand{\Span}{\mathrm{span}}
\newcommand{\Null}{\mathrm{null}}
\newcommand{\Rank}{\mathrm{rank}}
\newcommand{\mat}[1]{\begin{bmatrix}#1\end{bmatrix}}

%%%
% This is where the body of the document goes
%%%
\begin{document}
	\setheader{Math 211 (A01)}{Linear Algebra I}{Learning Objectives}{}{}{}

	\small
	\section*{Chapter 1}

	\begin{enumerate}
		\item[\bf Set Notation] 
			\begin{itemize}
				\item Given a set in set-builder notation, explicitly specify it by
					listing its elements or drawing it.
				\item Identify when one set is a subset of another.
				\item Compute and specify the result of an expression involving unions and intersections.
			\end{itemize}
		\item[\bf Vectors] 
			\begin{itemize}
				\item Add and scale vectors specified geometrically or component-wise.
				\item Write the definition of what it means for two vectors to be equal.
			\end{itemize}
		\item[\bf Linear Combinations] 
			\begin{itemize}
				\item Write the definition of a linear combination of two vectors.
				\item Compute linear combinations.
				\item Write a vector in $\R^2$ as a linear combination of other vectors in $\R^2$.
				\item Write the definition of span.
				\item Identify in simple cases (e.g. $\R^2$ or $\R^3$) if a vector is in 
					the span of others.
			\end{itemize}
		\item[\bf Linear Independence] 
			\begin{itemize}
				\item Write the definition of linear independence and linear dependence.
				\item Identify whether simple sets are linearly independent or dependent.
			\end{itemize}
		\item[\bf Subspace/Basis] 
			\begin{itemize}
				\item Write the definition of a subspace and basis.
				\item Produce a proof of whether or not a set (described in words or set notation) is
					a subspace.
				\item Identify an invalid proof of subspaceness.
				\item Define dimension.
			\end{itemize}
		\item[\bf Dot product/Length] 
			\begin{itemize}
				\item Write the definition of the dot product of two vectors written in components as well as 
					the geometric definition of the dot product involving lengths and angles.
				\item Write the definition of the length of a vector.
				\item Compute dot products and lengths.
				\item Use linearity of the dot product to compute other dot products.
				\item Find the angle between two vectors.
				\item Find the distance between two vectors.
				\item Define and produce unit vectors.
				\item Define orthogonality.
			\end{itemize}
		\item[\bf Projections] 
			\begin{itemize}
				\item Define projection.
				\item Project vectors onto lines and planes.
			\end{itemize}
		\item[\bf Lines \& Planes] 
			\begin{itemize}
				\item Specify lines and planes as spans or in vector form.
				\item Specify a given line or plane in scalar form.
			\end{itemize}
	\end{enumerate}

	\section*{Chapter 2}
	\begin{enumerate}
		\item[\bf SLEs] 
			\begin{itemize}
				\item Solve SLEs with unique solutions.
				\item Convert back and forth between a SLE and an augmented matrix.
				\item Given two solutions to a SLE, produce another.
				\item Recite a SLE has $0,1$, or $\infty$ solutions.
				\item Identify whether a system is inconsistent.
			\end{itemize}
		\item[\bf Row Reduction] 
			\begin{itemize}
				\item Reduce a matrix to rref.
				\item Identify whether a matrix is in rref.
				\item Produce the solution(s) to a SLE from rref form of the augmented matrix.
				\item List the elementary row operations.
			\end{itemize}
		\item[\bf Rank] 
			\begin{itemize}
				\item Define rank.
				\item Compute rank of a given matrix.
				\item Use rank to determine whether a system has a unique solution.
			\end{itemize}
		\item[\bf Spanning Sets] 
			\begin{itemize}
				\item Use row reduction to determine if a set of vectors is linearly independent.
				\item Use row reduction to find a basis for the span of a set of vectors.
				\item Use row reduction to determine if a vector is in the span of others.
				\item Use row reduction to write a vector as a linear combination of things in a basis.
			\end{itemize}
	\end{enumerate}
	
	\section*{Chapter 3}
	\begin{enumerate}
		\item[\bf Matrix Ops] 
			\begin{itemize}
				\item Index a matrix.
				\item Add and multiply matrices.
				\item Produce matrices where the order of multiplication does/doesn't matter.
				\item Rewrite a SLE as a matrix equation.
				\item Compute the transpose of a matrix.
				\item Write out the special matrices $I$ and $0$.
			\end{itemize}
		\item[\bf Matrix Inverses] 
			\begin{itemize}
				\item Define the inverse of a matrix.
				\item Compute the inverse of a matrix.
				\item Solve a matrix equation using an inverse.
				\item Relate rank and invertibility.
				\item Produce examples of invertible/non-invertible matrices.
				\item Prove the inversion formula for matrix products ($(AB)^{-1}=B^{-1}A^{-1}$).
			\end{itemize}
		\item[\bf Elementary Matrices] 
			\begin{itemize}
				\item Define an elementary matrix.
				\item Find the inverse of an elementary matrix.
				\item Decompose an invertible matrix as a product of elementary matrices.
			\end{itemize}
		\item[\bf Linear Transformations] 
			\begin{itemize}
				\item Write the definition of a linear transformation from $\R^n\to\R^m$.
				\item Compute the image of a vector under a linear transformation described in
					words or a matrix.
				\item Produce the standard matrix for a linear transformation described in some
					other way.
				\item Determine whether a standard matrix for a linear transformation will be 
					invertible based on the description of the linear transformation
					(and without actually trying to compute the inverse of a matrix).
				\item Define and compute the image/range and null space/kernel of a linear transformation.
				\item Define and compute the row/column space of a matrix.
				\item Determine whether or not a given transformation is linear.
				\item State the rank-nullity theorem.
				\item Use the rank-nullity theorem to determine the dimension of the image and
					kernel of a given linear transformation.
			\end{itemize}
	\end{enumerate}
	
	\section*{Chapter 4}
	\begin{enumerate}
		\item[\bf Change of Basis] 
			\begin{itemize}
				\item Write a vector given in the standard basis in another basis.
				\item Write a linear transformation in a different basis.
			\end{itemize}
	\end{enumerate}
	\section*{Chapter 5}
	\begin{enumerate}
		\item[\bf Determinants] 
			\begin{itemize}
				\item Define the determinant as an oriented volume.
				\item Relate the determinant to invertibility.
				\item Compute the determinant of a $2\times 2$, $3\times 3$, or 
					sparse $n\times n$ matrix.
				\item Use the multiplicativity, inverse property, and transpose property
					of the determinant to compute the determinant of a composition of
					matrices.
				\item Use determinants to compute volumes.
				\item Compute the determinant of particular linear transformations (those with
					nice geometric descriptions) without using a matrix.
			\end{itemize}
	\end{enumerate}
	\section*{Chapter 6}
	\begin{enumerate}
		\item[\bf Eigen Vectors/Values] 
			\begin{itemize}
				\item Define eigenvectors/values.
				\item Compute eigenvectors/values.
				\item Relate the set of eigenvalues of a particular matrix to its determinant.
				\item Define and compute the characteristic polynomial of a matrix.
				\item Create a matrix with given eigenvalues/eigenvectors.
				\item Diagonalize a matrix.
				\item Define eigen space.
				\item Use diagonalization to compute large powers of a matrix.
			\end{itemize}
	\end{enumerate}
	\section*{Chapter 7}
	\begin{enumerate}
		\item[\bf Orthogonality] 
			\begin{itemize}
				\item Use Gram-Schmidt to produce an orthonormal basis for a subspace.
			\end{itemize}
	\end{enumerate}
\end{document}
