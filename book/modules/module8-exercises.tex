\begin{exercises}

	\begin{problist}
		\prob
		\begin{enumerate}
			\item Let $\vec u=\xhat+8\yhat$, $\vec v=-\xhat+ 3\yhat$,
				and $\vec w=2\xhat$.


				\begin{enumerate}
					\item Find $[\vec u]_{\mathcal E}$,
						$[\vec v]_{\mathcal E}$ and
						$[\vec w]_{\mathcal E}$, where $\mathcal
						E$ is the standard basis for $\R^{2}$.

					\item Let $\mathcal A=\Set{3\xhat+2\yhat,4\xhat-\yhat}$.
						Find $[\vec u]_{\mathcal A}$, $[\vec
						v]_{\mathcal A}$ and $[\vec w]_{\mathcal
						A}$.

					\item Let $\mathcal B=\Set{11\yhat,\xhat+\frac{5}{2}\yhat}$.
						Find $[\vec u]_{\mathcal B}$, $[\vec
						v]_{\mathcal B}$ and $[\vec w]_{\mathcal
						B}$.
				\end{enumerate}

			\item Let $\vec q=11\yhat-4\zhat$,
				$\vec r=5\xhat-12\yhat+8\zhat$, and $\vec s=\xhat-5\yhat+2\zhat$.

				\begin{enumerate}
					\item Find $[\vec q]_{\mathcal E}$,
						$[\vec r]_{\mathcal E}$ and
						$[\vec s]_{\mathcal E}$ where
						$\mathcal E$ is the standard
						basis for $\R^{3}$.

					\item Let $\mathcal D=\Set{\xhat+2\yhat,-3\xhat+5\yhat-4\zhat,-8\xhat+4\yhat+11\zhat}$.
						Find $[\vec q]_{\mathcal D}$, $[\vec
						r]_{\mathcal D}$ and $[\vec s]_{\mathcal
						D}$.

					\item Let $\mathcal F=\Set{\xhat+4\yhat+4\zhat,-3\xhat+20\yhat,21\yhat+16\zhat}$.
						Find $[\vec q]_{\mathcal F}$, $[\vec
						r]_{\mathcal F}$ and $[\vec s]_{\mathcal
						F}$.
				\end{enumerate}
		\end{enumerate}
		\begin{solution}
			\begin{enumerate}
				\item 
					\begin{enumerate}
						\item $[\vec{u}]_{\mathcal{E}} = \mat{1\\8}$,
							$[\vec{v}]_{\mathcal{E}} = \mat{-1\\3}$, and
							$[\vec{w}]_{\mathcal{E}} = \mat{2\\0}$.
						\item $[\vec{u}]_{\mathcal{A}} = \mat{3\\-2}$,
							$[\vec{v}]_{\mathcal{A}} = \mat{1\\-1}$, and
							$[\vec{w}]_{\mathcal{A}} = \mat{2/11\\4/11}$.
						\item $[\vec{u}]_{\mathcal{B}} = \mat{1/2\\1}$,
							$[\vec{v}]_{\mathcal{B}} = \mat{1/2\\-1}$, and
							$[\vec{w}]_{\mathcal{B}} = \mat{-5/11\\2}$.
					\end{enumerate}
				\item
					\begin{enumerate}
						\item $[\vec{q}]_{\mathcal{E}} = \mat{0\\11\\-4}$,
							$[\vec{r}]_{\mathcal{E}} = \mat{5\\-12 \\8}$, and
							$[\vec{s}]_{\mathcal{E}} = \mat{1\\-5\\2}$.
						\item We are given $\mathcal{D} = \Set*{\mat{1\\2\\0}_{\mathcal{E}}, \mat{-3\\5\\-4}_{\mathcal{E}}, 
							\mat{-8\\4\\11}_{\mathcal{E}}}$.
							To find $[\vec{q}]_{\mathcal{D}}$, we need to find scalars 
							$x, y, z$ such that
							\[
								x \mat{1\\2\\0} + y \mat{-3\\5\\-4} +z 
								\mat{-8\\4\\11} = \mat{0\\11\\-4},
							\]
							which gives rise to a system of equations whose augmented matrix is 
							\[
								\begin{bmatrix}[ccc|c]
									1 & -3 & -8 & 0\\
									2 & 5 & 4 & 11\\
									0 & -4 & 11 & -4\\
								\end{bmatrix}.
							\]
							This can be row reduced to
							\[
								\begin{bmatrix}[ccc|c]
									1 & 0 & 0 & 3\\
									0 & 1 & 0 & 1\\
									0 & 0 & 1 & 0\\
								\end{bmatrix}.
							\]
							Therefore, $[\vec{q}]_{\mathcal{D}} = \mat{3\\1\\0}$. Similarly we can find 
							\[
								[\vec{r}]_{\mathcal{D}} = \mat{-1\\-2\\0},\quad [\vec{s}]_{\mathcal{D}} = \mat{-66/67\\-39/67\\-2/67}.
							\]
						\item We are given $\mathcal{F} = \Set*{ \mat{1\\4\\4}_{\mathcal{E}}, \mat{-3\\20\\0}_{\mathcal{E}}, 
							\mat{0\\21\\16}_{\mathcal{E}}}$. Therefore,
							$[\vec{q}]_{\mathcal{F}} = \mat{3 \\ 1 \\ -1}$,
							$[\vec{r}]_{\mathcal{F}} = \mat{2 \\ -1 \\0}$,
							and $[\vec{s}]_{\mathcal{F}} = \mat{-23/130\\-51/130\\11/65}$.
					\end{enumerate}
			\end{enumerate}
		\end{solution}

		\prob
		\begin{enumerate}
			\item Let $[\vec a]_{\mathcal E}=\mat{5\\-12}$ where $\mathcal
				E$ is the standard basis for $\R^{2}$. Find a
				basis $\mathcal M$ for $\R^{2}$ such that
				$[\vec a]_{\mathcal M}=\mat{1\\0}$.

			\item Let $[\vec b]_{\mathcal E}=\mat{2\\1\\0}$ where $\mathcal
				E$ is the standard basis for $\R^{3}$. Find a
				basis $\mathcal N$ for $\R^{3}$ such that
				$[\vec b]_{\mathcal N}=\mat{0\\1\\2}$.
		\end{enumerate}
		\begin{solution}
			\begin{enumerate}
				\item Let $\mathcal{M} = \Set{ \vec{u}, \vec{v}}$. 
					Then $[\vec{a}]_{\mathcal{M}} = \mat{1\\0} \Rightarrow \vec{a} = 1 \vec{u} + 0 
					\vec{v} \Rightarrow \vec{a} = \vec{u}$. So as long as we choose the first vector in $\mathcal{M}$ as $\vec{a}$, any 
					linearly independent second vector will do. Thus take $\mathcal{M} = \Set{5\xhat - 12 \yhat , \xhat }$
				\item The observation here is that the numbers that appear in the coordinates in both $[\vec{b}]_{\mathcal{E}}$ and 
					$[\vec{b}]_{\mathcal{N}}$ are same but they are just shuffled. Thus we can take $\mathcal{N}$ to be the corresponding 
					permutation of  $\mathcal{E}$. Take $\mathcal{N} = \Set{ \zhat, \yhat, \xhat}$
			\end{enumerate}
		\end{solution}

		\prob Determine the orientation of each of the following bases
		for $\R^{2}$.
		\begin{enumerate}
			\item $\Set*{\mat{-5\\0},\mat{0\\-2}}$

			\item $\Set*{\mat{2\\-6},\mat{-4\\1}}$

			\item $\Set*{\mat{1\\2},\mat{2\\1}}$

			\item $\Set*{\mat{6\\-1},\mat{2\\3}}$
		\end{enumerate}
		\begin{solution}
			\begin{enumerate}
				\item Positively oriented. (Rotate both vectors 180 degrees clockwise and scale)
				\item Negatively oriented.
				\item Negatively oriented. ( $\mat{1\\2}$ is to be transformed to $\vec{e}_1$. If we rotate it clockwise at some stage it points
					 at the same direction as $\mat{2\\1}$, thus making it linearly dependent. If we try to rotate it counter clockwise, at
					 some stage it also points at the same (negative) direction as $\mat{2\\1}$, where it becomes linerly dependent. )
				\item Positively oriented.
			\end{enumerate}
		\end{solution}

		\prob
		\begin{enumerate}
			\item Determine the orientation of the basis $\mathcal V
				= \Set*{\vec v_1, \vec v_2, \vec v_3}$ for $\R^{3}$
				where $\vec v_{1} = \mat{1\\0\\0}$,
				$\vec v_{2} = \mat{0\\1\\0}$, and $\vec v_{3} =
				\mat{0\\0\\-1}$.

			\item Consider the basis $\mathcal V' = \Set*{\vec v_1,
				3\vec v_2, \vec v_3}$ for $\R^{3}$. What is the
				orientation of this basis?

			\item Consider the basis $\mathcal V'' = \Set*{-4\vec v_1,
				\vec v_2, \vec v_3}$ for $\R^{3}$. What is the
				orientation of this basis?
		\end{enumerate}
		\begin{solution}
			\begin{enumerate}
				\item Negatively oriented. ( only $\vec{v}_3$ needs to be taken care of. To transform it to $\vec{e}_3$, no matter what we do,
					at some stage we have to cross the $xy$-plane. At that stage it becomes linearly dependent.)
				\item Negatively oriented. ( $3\vec{v}_2$ can be scaled to $\vec{v}_2$. Then we are back at situation (a).)
				\item Positively oriented. (This is a little bit tricky. Note that $\vec{v}_2$ is already $\vec{e}_2$. So we can leave that 
					alone and limit our operations to the $xz$-plane. In the $xz$-plane rotate $-4 \vec{v}_1$ and $\vec{v_3}$ counter 
					clockwise 180 degrees. Then $-4 \vec{v}_1$ transforms to $4 \vec{e}_1$, which then can be scaled to $\vec{e}_1$, and 
					$\vec{v}_3$ transforms to $\vec{e}_3$.)
			\end{enumerate}
		\end{solution}

		\prob
		\begin{enumerate}
			\item Give two examples of positively oriented bases for
				$\R^{2}$ and briefly explain how you know their orientation.

			\item Give two examples of negatively oriented bases for
				$\R^{2}$ and briefly explain how you know their orientation.

			\item How does negating one vector in a basis change the
				orientation?

			\item How does swapping the order of two different vectors
				in a basis change the orientation?
		\end{enumerate}
	\end{problist}
\end{exercises}
