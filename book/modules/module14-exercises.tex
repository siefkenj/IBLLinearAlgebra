\begin{exercises}
	\begin{problist}
		\prob Let $\mathcal{T}:\R^{2}\to\R^{2}$ be defined by $\mathcal{T}\mat{x\\y}
		=\matc{3x-y\\x-\tfrac{1}{4}y}$. Find the volume of $\mathcal{T}(C_{2})$.
		\begin{solution}
			The volume of $\mathcal T(C_{2})$ is equal to the absolute value of the
			determinant of $\mathcal T$. We have that
			\[
				[\mathcal T]_{\mathcal E} =
				\mat{
					3 & - 1  \\
					1 & -1/4
				},
			\] so $\det \mathcal T = -3/4 + 1 = 1/4$. Since this number is positive, it is also the desired volume.
		\end{solution}

		\prob Let $\mathcal{S}:\R^{3}\to\R^{3}$ be defined by
		$\mathcal{S}\mat{x\\y\\z}=\matc{2x+y+z\\x-\tfrac{1}{2}y\\z}$. Find the volume
		of $\mathcal{S}(C_{3})$.
		\begin{solution}
			We start by computing the determinant of $\mathcal S$.
			The determinant of $\mathcal S$ can be computed from $[\mathcal S]_{\mathcal E}$, which is
			given by
			\[
				[\mathcal S]_{\mathcal E} =
				\mat{
					2 & 1    & 1 \\
					1 & -1/2 & 0 \\
					0 & 0    & 1
				}.
			\] Since determinant is preserved by row operations of the form ``add a multiple of
			one row to another'', we can partially row reduce $[\mathcal S]_{\mathcal E}$ (using
			only that row operation) without changing the determinant. Thus, the determinant
			of $[\mathcal S]_{\mathcal E}$ is the same as the determinant of
			\[
				\mat{
					2 & 1  & 1    \\
					0 & -1 & -1/2 \\
					0 & 0  & 1
				}.
			\]
			This matrix is triangular, so the determinant is just the product of the
			entries on the diagonal. Therefore, $\det [\mathcal S]_{\mathcal E} = -2$. But volume is non-negative,
			so the volume of $\mathcal S(C_3)$ is $2$.
		\end{solution}

		\prob Let $\mathcal{T}:\R^{2}\to\R^{2}$ be defined by $\mathcal{T}\mat{x\\y}
		=\matc{x+2y\\-x-y}$.
		\begin{enumerate}
			\item Draw $\mathcal{E}$ and $\mathcal{T}(\mathcal{E})$ and then determine
				whether $\mathcal{T}$ is orientation preserving or orientation reversing.

			\item Find $\det(\mathcal{T})$.
		\end{enumerate}
		\begin{solution}
			\begin{enumerate}
				\item Computing, we see $\mathcal T(\xhat) = \mat{1\\-1}$ and
					$\mathcal T(\yhat)=\mat{2\\-1}$. Drawing these two vectors, we see
					that $\mathcal T(\xhat),\mathcal T(\yhat)$
					can be continuously transformed back into $\xhat,\yhat$ while staying
					linearly independent the whole time. Therefore $\mathcal T$ is orientation
					preserving.

				\item $\det \mathcal T$ is equal to the determinant of the matrix
					\[
						[\mathcal T]_{\mathcal E}=\mat{
							1  & 2  \\
							-1 & -1
						},
					\]
					which is $1$.
			\end{enumerate}
		\end{solution}

		\prob For each linear transformation defined below, find its determinant.
		\begin{enumerate}
			\item $\mathcal{S}:\R^{2}\to\R^{2}$, where $\mathcal{S}$ shortens
				every vector by a factor of $\tfrac{2}{3}$.

			\item $\mathcal{R}:\R^{2}\to\R^{2}$, where $\mathcal{R}$ is rotation
				counter-clockwise by $90^{\circ}$.

			\item $\mathcal{F}:\R^{2}\to\R^{2}$, where $\mathcal{F}$ is
				reflection across the line $y=-x$.

			\item $\mathcal{G}:\R^{2}\to\R^{2}$, where
				$\mathcal{G}(\vec x)=\mathcal{P}(\vec x)+ \mathcal{Q}(\vec x)$
				and where $\mathcal{P}$ is projection onto the line $y=x$ and
				$\mathcal{Q}$ is projection onto the line $y=-\tfrac{1}{2}x$.

			\item $\mathcal{T}:\R^{3}\to\R^{3}$, where
				$\mathcal{T}\mat{x\\y\\z}=\matc{x-y+z\\z+x-\tfrac{1}{3}y\\z}$.

			\item $\mathcal{J}:\R^{3}\to\R^{3}$, where
				$\mathcal{J}\mat{x\\y\\z}=\matc{0\\0\\x+y+z}$.

			\item $\mathcal{K}\circ \mathcal{H}:\R^{2}\to\R^{2}$, where
				$\mathcal{H}\mat{x\\y}=\matc{x+2y\\-x-y}$,
				and $\mathcal{K}\mat{x\\y}=\matc{-x-2y\\x+y}$.
		\end{enumerate}
		\begin{solution}
			\begin{enumerate}
				\item The matrix for $\mathcal S$ in any basis is $
					\mat{
						2/3 & 0 \\
						0 & 2/3
					}$, so the determinant is $4/9$.

				\item $\mathcal R$ does not change volume or orientation so its determinant is $1$.

				\item $\mathcal F$ does not change volume but it reverses orientation so its determinant
					is $-1$.
				
				\item Though the determinants of $\mathcal P$ and $\mathcal Q$ are both $0$, the determinant
					of $\mathcal G$ is not zero! We can compute the standard matrix for $\mathcal G$ by
					noticing $\mathcal G(\xhat)=\mat{13/10\\1/10}$ and $\mathcal G(\yhat)=\mat{1/10\\7/10}$.
					Therefore
					\[
						[\mathcal G]_{\mathcal E} =\mat{13/10&1/10\\1/10&7/10}
					\]
					and so $\det\mathcal G=9/10$.
				
				\item The matrix $[\mathcal T]_{\mathcal E}$ is given by
					\[
						[\mathcal T]_{\mathcal E} =
						\mat{
							1 & -1   & 1 \\
							1 & -1/3 & 1 \\
							0 & 0    & 1
						}.
					\] Subtracting the first row from the second gives the matrix
					\[
						\mat{
							1 & -1  & 1 \\
							0 & 2/3 & 0 \\
							0 & 0   & 1
						},
					\] which has the same determinant as $[\mathcal T]_{\mathcal E}$, and since this
					matrix is upper triangular, its determinant is simply $2/3$.

				\item The map $\mathcal J$ maps every vector in $\R^{3}$ into $\Span\Set*{\mat{
						0 \\
						0 \\
						1}}$, hence $\mathcal J$ is not invertible. Therefore,
					$\det \mathcal J = 0$.

				\item The determinant of the composition of the two maps is just the
					product of the determinants of the two maps. The matrices 
					for $\mathcal K$ and $\mathcal H$ (with respect
					to $\mathcal E$) are
					\[
						\mat{
							1  & 2  \\
							-1 & -1
						}\qquad\text{and}\qquad
						\mat{
							-1 & -2 \\
							1  & 1
						}
					\] and each has determinant one, so the determinant of $\mathcal K\circ\mathcal H$
					is also $1$.
			\end{enumerate}
		\end{solution}

		\prob Let $A=\mat{2&3\\1&5}$.
		\begin{enumerate}
			\item Use elementary matrices to find $\det(A)$.

			\item Draw a picture of the parallelogram given by the rows of $A$.
				\label{PROBMOD14-rows}

			\item Draw a picture of the parallelogram given by the columns of $A$.
				\label{PROBMOD14-cols}

			\item How do the areas of the parallelograms drawn in parts \ref{PROBMOD14-rows}
				and \ref{PROBMOD14-cols} relate?
		\end{enumerate}
		\begin{solution}

			\begin{enumerate}
				\item Put $E_{1} =
					\mat{
						1    & 0 \\
						-1/2 & 1
					}$. Then
					\[
						E_{1} A =
						\mat{
							2 & 3   \\
							0 & 7/2
						}.
					\] Put $E_{2} =
					\mat{
						1 & 0   \\
						0 & 2/7
					}$. Then
					\[
						E_{2} E_{1} A =
						\mat{
							2 & 3 \\
							0 & 1
						}.
					\] Put $E_{3} =
					\mat{
						1 & -3 \\
						0 & 1
					}$. Then
					\[
						E_{3} E_{2} E_{1} A =
						\mat{
							2 & 0 \\
							0 & 1
						}.
					\] Finally, put $E_{4} =
					\mat{
						1/2 & 0 \\
						0   & 1
					}$. Then
					\[
						E_{4} E_{3} E_{2} E_{1} A =
						\mat{
							1 & 0 \\
							0 & 1
						}.
					\] Therefore,
					\[
						\det A = \det E_{1}^{-1}\det E_{2}^{-1}\det E_{3}^{-1}\det E_{4}
						^{-1}= 7.
					\]

				\item

				\item

				\item They have the same area.
			\end{enumerate}
		\end{solution}

		\prob Let $A=\mat{1&2&0\\0&2&1\\1&2&3}$.
		\begin{enumerate}
			\item \label{Module14-q8} Use elementary matrices to find $\det(A)$.

			\item Find $\det(A^{-1})$.

			\item Find $\det(A^{T})$, and compare your answer with
				\ref{Module14-q8}. Are they the same? Explain.
		\end{enumerate}
		\begin{solution}
			\begin{enumerate}
				\item By row reducing and keeping track
					of our steps, we see that
					\[
						E_5E_4E_3E_2E_1 A = I
					\]
					where
					\[
						E_{1} =
						\mat{
							1  & 0 & 0 \\
							0  & 1 & 0 \\
							-1 & 0 & 1
						}\qquad
						E_{2} =
						\mat{
							1  & 0 & 0 \\
							0  & 1/2 & 0 \\
							0 & 0 & 1
						}
					\]
					\[
						E_{3} =
						\mat{
							1  & 0 & 0 \\
							0  & 1 & 0 \\
							0 & 0 & 1/3
						}\qquad
						E_{4} =
						\mat{
							1  & -2 & 0 \\
							0  & 1 & 0 \\
							0 & 0 & 1
						}
					\]
					\[
						E_{5} =
						\mat{
							1  & 0 & 0 \\
							0  & 1 & -1 \\
							0 & 0 & 1
						}
					\]
					Therefore $A=E_1^{-1}E_2^{-1}E_3^{-1}E_4^{-1}E_5^{-1}$.
					By thinking about the relationship between elementary
					matrices and determinants, we see that $\det E_1^{-1}
					=\det E_4^{-1} =\det E_5^{-1} = 1$ and
					that $\det E_2^{-1} = 2$ and $\det E_3^{-1} = 3$.
					Therefore $\det A = 6$.

				\item $\det(A^{-1}) = 1/\det(A) = 1/6$.

				\item We have
					\[
						A^{T} =
						\mat{
							1 & 0 & 1 \\
							2 & 2 & 2 \\
							0 & 1 & 3
						}.
					\] Note that
					\[
						(E_{1} A)^{T} = A^{T} E_{1}^{T} =
						\mat{
							1 & 0 & 0 \\
							2 & 2 & 0 \\
							0 & 1 & 3
						}.
					\] This matrix is \emph{lower} triangular, so the determinant is
					equal to the product of the diagonal entries, which is still $6$.
					Further $\det(E_{1}^{T}) = 1$, and so $\det(A^{T}) = 6$.
			\end{enumerate}
		\end{solution}

		\prob Let $A$ be an $n \times n$ matrix that can be decomposed into the
		product of elementary matrices.
		\begin{enumerate}
			\item What is $\Rank(A)$? Justify your answer.

			\item What is $\Null(A^{-1})$? Justify your answer.
		\end{enumerate}
		\begin{solution}
			\begin{enumerate}
				\item The rank of $A$ is equal to $n$, since each elementary matrix
					has non-zero determinant and since $A$ can be expressed as a product
					of elementary matrices, it also has non-zero determinant.

				\item The nullspace of $A^{-1}$ is trivial (i.e. equal to $\Set{\vec 0}$), since $A^{-1}$ is
					invertible.
			\end{enumerate}
		\end{solution}

		\prob Anna and Ella are studying the relationship between determinant and
		volume. In particular, they are studying $\mathcal{S}:\mathbb{R}^{3}\rightarrow
		\mathbb{R}^{3}$ defined by $\mathcal{S}\mat{x \\ y \\ z}=\mat{4x \\ 2z \\ 0}$,
		and $\mathcal{T}:\mathbb{R}^{3}\rightarrow\mathbb{R}^{2}$ defined by
		$\mathcal{T}\mat{x \\ y \\ z}=\mat{2x \\ 8z}$.

		For each conversation below, (a) evaluate Anna and Ella's arguments as \emph{correct},
		\emph{mostly correct}, or \emph{incorrect}; (b) point out where each argument
		makes correct/incorrect statements; (c) give a correct numerical value
		for the determinant or explain why it doesn't exist.
		\begin{enumerate}
			\item \emph{Anna says:}

				Since the image of $C_{3}$ under $\mathcal{S}$ is the
				parallelepiped generated by
				$\mat{4 \\ 0 \\ 0},\mat{0 \\ 0 \\ 0}, and \mat{0 \\ 2 \\ 0}$, which
				is 2-dimensional parallelogram, the volume of
				$\mathcal{S}(C_{3})$ is just the area of this parallelogram, which
				is 8. Thus, $\det(\mathcal{S})=8$.

				\emph{Ella says:}

				$\det(\mathcal{S})$ is undefined, because $\mathcal{S}$ is not
				invertible.

			\item \emph{Anna says:}

				Since the image of $C_{3}$ under $\mathcal{T}$ is the
				parallelepiped generated by $\mat{2 \\ 0}$, $\mat{0 \\ 0}$, and
				$\mat{0 \\ 8}$, which is a parallelogram in $\mathbb{R}^{2}$,
				the signed volume of $\mathcal{T}(C_{3})$ is just the signed
				area of this parallelogram, which is 16. Thus,
				$\det(\mathcal{T})=16$.

				\emph{Ella says:}

				$\det(\mathcal{T})$ is undefined, because $\det(\mathcal{T})$ is
				only defined when the domain and codomain of $\mathcal{T}$ are
				the same.
		\end{enumerate}
		\begin{solution}
			\begin{enumerate}
				\item \emph{Anna's argument is incorrect.}

					\emph{Reason:} Since $\mathcal{S}$ is a linear
					transformation on $\R^{3}$, its determinant is given by the signed
					change of \emph{3-dimensional volume}. Anna's argument is incorrect
					because she considered the 2-dimensional volume of $\mathcal{S}
					(C_{3})$.

					\emph{Ella's argument is incorrect.}

					\emph{Reason:} The determinant is defined for all linear
					transformations from $\R^{n}$ to $\R^{n}$, no matter whether
					it is invertible or not.

					Finally, $\det(\mathcal{S})=0$, because since $\mathcal{S}(C_{3}
					)$ is a \emph{2-dimensional} object in $\R^{3}$, its \emph{3-dimensional
					volume} is $0$. Therefore, $\VolChange(\mathcal{S})=0$, and we
					conclude that $\det(\mathcal{S})=0$.

				\item \emph{Anna's argument is incorrect.}

					\emph{Reason:} The determinant function is only defined for
					linear transformations with same domain and codomain.

					\emph{Ella's argument is correct.}

					Finally, $\det(\mathcal{T})$ is undefined, because the domain
					and codomain of $\mathcal{T}$ are not the same.
			\end{enumerate}
		\end{solution}
	\end{problist}
\end{exercises}
