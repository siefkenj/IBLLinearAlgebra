\begin{center}
	{\color{myorange}\huge\bfseries\sffamily Linear Algebra}\\

\vspace{.2in}
{
\it \copyright\,Jason Siefken, 2016--2022 \\
Creative Commons By-Attribution Share-Alike\, \makebox(30,5){\includegraphics[height=1.2em]{by-sa.pdf}}
}
\end{center}

\section*{About this Book}

\subsection*{For the student}

This book is your introductory guide to linear algebra. It is divided into
\emph{modules}, and each module is further divided into \emph{exposition},
\emph{practice problems}, and \emph{core exercises}.

The \emph{exposition} is easy to find---it's the text that starts each
module and explains the big ideas of linear algebra.  The \emph{practice
problems} immediately follow the exposition and are there so you can
practice with concepts you've learned.  Following the practice problems
are the \emph{core exercises}. The core exercises build up, through
examples, the concepts discussed in the exposition.

To optimally learn from this text, you should:
\begin{itemize}
	\item Start each module by reading through the \emph{exposition} to get familiar with the main ideas and 
		linear algebra terminology.

	\item Work through the \emph{core exercises} to develop an understanding and intuition behind the main ideas
		and their subtleties.

	\item Re-read the \emph{exposition} and identify which concepts each core exercise connects with.

	\item Work through the \emph{practice problems}. These will serve as a check on whether you've understood
		the main ideas well enough to apply them.
\end{itemize}

{\bf The core exercises.} Most (but not all) core exercises will be
worked through during lecture time, and there is space for you to work
provided after each
of the core exercises. The point of the core exercises is to develop the main ideas of
linear algebra by exploring examples. When working on core exercises, think
``it's the journey that matters not the destination''. The
answers are not the point! If you're struggling, keep with it. The
concepts you struggle with you remember well, and if you look up the
answer, you're likely to forget just a few minutes later. 

{\bf So many definitions.} A big part of linear algebra is learning precise and
technical language\footnote{ Beyond three dimensions, things get very confusing
very quickly.  Having precise definitions allows us to make arguments that
rely on logic instead of intuition; and logic works in all dimensions.}.
There are many terms and definitions you need to learn, and by far the
best way to successfully learn these terms is to understand where they
come from, why they're needed, and practice using them. That is, don't
try to memorize definitions word for word. Instead memorize the idea
and \emph{reconstruct} the definition; go through the core exercises and
identify which definitions appear where; and explain linear algebra to
others using these technical terms.

{\bf Contributing to the book.} Did you find an error? Do you
have a better way to explain a linear algebra concept? Please,
contribute to this book!  This book is open-source, and we welcome
contributions and improvements. To contribute to/fix part of
this book, make a \emph{Pull Request} or open an \emph{Issue} at
\url{https://github.com/siefkenj/IBLLinearAlgebra}. If you contribute,
you'll get your name added to the contributor list.


\subsection*{For the instructor}

This book is designed for a one-semester introductory linear algebra course
course with a focus on geometry (MAT223 at the University of Toronto). 
It has not been designed for an ``intro to proofs''-style course, but could be adapted for one.

Unlike a traditional textbook that is grouped into chapters and sections
by subject, this book is grouped into modules. Each module contains exposition
about a subject, practice problems (for students to work on by themselves), and core exercises
(for students to work on with your guidance). Modules group related concepts, but the 
modules have been designed to facilitate learning linear algebra rather than to serve
as a reference. For example, information about change-of-basis is spread across several non-consecutive
modules; each time change-of-basis is readdressed, more detail is added.

{\bf Using the book.} This book has been designed for use in large 
active-learning classrooms driven by a \emph{think, pair-share}/small-group-discussion format.
Specifically, the \emph{core exercises} (these are the problems which aren't labeled ``Practice Problems''
and for which space is provided to write answers) are designed for use during class time.

A typical class day looks like:
\begin{enumerate}
	\item {\bf Student pre-reading.} Before class, students will read through the relevant module.

	\item {\bf Introduction by instructor.} This may involve giving a definition,
		a broader context for the day's topics, or answering questions.

	\item {\bf Students work on problems.} Students work individually or in pairs/small groups
		on the prescribed core exercise. During this time the instructor moves
		around the room addressing questions that students may have and giving
		one-on-one coaching.

	\item {\bf Instructor intervention.} When most students have successfully solved
		the problem, the instructor refocuses the class by providing an
		explanation or soliciting explanations from students.
		This is also time for the instructor to ensure that everyone has
		understood the main point of the exercise (since it is sometimes
		easy to miss the point!).

		If students are having trouble, the instructor can give hints
		and additional guidance to ensure students' struggle is productive.

	\item {\bf Repeat step 3.}
\end{enumerate}

Using this format, students are thinking (and happily so) most of the class. Further,
after struggling with a question, students are especially primed to hear the insights of the instructor.

{\bf Conceptual lean.}
The \emph{core exercises} are geared towards concepts instead of computation, though some core exercises
focus on simple computation. They also have a geometric lean. Vectors are initially
introduced with familiar coordinate notation, but eventually, coordinates are understood to be
\emph{representations} of vectors rather than ``true'' geometric vectors, and objects like the
determinant are defined via oriented volumes rather than formulas involving matrix entries.

Specifically lacking are exercises focusing on the mechanical skills of row reduction and
computing matrix inverses. Students must practice these skills, but they require little instructor
intervention and so can be learned outside of lecture (which is why core exercises don't focus on
these skills).

{\bf How to prepare.}
Running an active-learning classroom is less scripted than lecturing.
The largest
challenges are: (i) understanding where students are at, (ii) figuring out what to do given the current
understanding of the students, and (iii) timing.

To prepare for a class day, you should:
\begin{enumerate}
	\item {\bf Strategize about learning objectives.} Figure out what the point of the day's lesson is
		and brain storm some examples that would illustrate that point.
	\item {\bf Work through the core exercises.} 
	%	By working through the exercises yourself, you
	%	will be ready to build off student reasoning, and better able to direct a class towards
	%	the important ideas\footnote{ The content of linear algebra is fairly non-linear. One of the hardest parts
	%	of teaching linear algebra is coming up with an explanation that only depends on ideas that have already been taught.}.
	\item {\bf Reflect.} Reflect on how each core exercise addresses the day's goals. Compare with the examples you
		brainstormed and prepare follow-up questions that you can use in class to test for understanding.
	\item {\bf Schedule.} Write timestamps next to each core exercise indicating at what minute you hope
		to start each exercise. Give more time for the exercises that you judge as foundational, and be prepared
		to triage. It's appropriate to leave exercises or parts of exercises for homework, but change the order
		of exercises at your peril---they really do build on each other.
\end{enumerate}

A typical
50 minute class is enough to get through 2--3 core exercises (depending on the difficulty), and class observations
show that class time is split 50/50 between students working and instructor explanations.

\subsection*{License}
 Unless otherwise mentioned, pages of this document are licensed under
the Creative Commons By-Attribution Share-Alike License. That means, you are free
to use, copy, and modify this document provided that you provide attribution to the
previous copyright holders and you release your derivative work under the same license.
Full text of the license is at \url{http://creativecommons.org/licenses/by-sa/4.0/}

If you modify this document, you may add your name to the copyright list. Also,
if you think your contributions would be helpful to others, consider making a
pull request, or opening an \emph{issue} at \url{https://github.com/siefkenj/IBLLinearAlgebra}

{\bf Incorporated content.}
Content from other sources is reproduced here with permission and retains the
Author's copyright. Please see the footnote of each page to verify the
copyright.

Included in this text are tasks created by the Inquiry-Oriented Linear Algebra (IOLA) project. Details
about these tasks can be found on their website \url{http://iola.math.vt.edu/}. Also included are some
practice problems from Beezer's \emph{A First Course in Linear Algebra} (marked with the symbol \beezer next to the
problem), and from Hefferon's \emph{Linear Algebra} (marked with the symbol \hefferon next to the problem).

{\bf Contributing.} You can report errors in the book or contribute to the book by filing an \emph{Issue} or
a \emph{Pull Request} on the book's GitHub page: \url{https://github.com/siefkenj/IBLLinearAlgebra/}

