		\begin{enumerate}
			\item
					The rank of a matrix $A$ is the number of pivots in rref($A$);
					the rank of a linear transformation $\mathcal T$ is the dimension of the range of $\mathcal T$.
			\item
			\begin{enumerate}
				\item $A=\begin{bmatrix}
					1 & 0 & 0 \\
					0 & 0 & 0
					\end{bmatrix}$ is one example. There are many others with linearly dependent rows, not both zero.
				\item $A=\begin{bmatrix}
					1 & 0 & 0 \\
					0 & 1 & 0
					\end{bmatrix}$ is one example. There are many others with linearly independent rows.
				\item Such a matrix does not exist. $A$ only has two rows, so rref($A$) can have at most two pivot positions. Equivalently, the columnspace of $A$ is a subspace of $\R^2$, so it can't be $3$ dimensional.
				\item $A=\begin{bmatrix}
					0 & 0 & 0 \\
					0 & 0 & 0
					\end{bmatrix}$ is the only example. There are no others.
			\end{enumerate}
			\item A linear transformation $\mathcal L:\R^n\to\R^m$
				is one-to-one if its nullity is $0$, and it is onto if its range is
				all of $\R^m$.
			\begin{enumerate}
				\item Using the rank-nullity theorem, we have rank$(\mathcal L) = n$.
				\item Similarly, if $\mathcal L$ is not one-to-one, we have rank$(\mathcal L) < n$. Note that we can never have rank$(\mathcal L)>n$ (why?).
				\item $\mathcal L$ is onto if rank$(\mathcal L)=m$.
				\item $\mathcal L$ is not onto if rank$(\mathcal L)<m$.
			\end{enumerate}
			\item
			\begin{enumerate}
				\item Since the ranks of $\mathcal T$ and $\mathcal S$ are both $2$,
					their ranges are both planes in $\R^{3}$. Let $\mathcal T=\mathcal
					S$ be projection onto the $xy$-plane. Then,
					$\mathcal S=\mathcal S\circ \mathcal T$ is rank $2$.

				\item Again, the ranges of $\mathcal T$ and $\mathcal S$ must be planes,
					but this time we want the range of $\mathcal S \circ \mathcal T$
					to be a line. Let $\mathcal T$ be projection onto the $xy$-plane
					and let $\mathcal S$ be projection onto the $xz$-plane. Then
					$\mathcal S\circ \mathcal T$ is projection onto the $x$-axis, which
					has rank $1$.

				\item This is impossible. Since $\Range(\mathcal S\circ \mathcal T)
					\subseteq$ $\Range(\mathcal S)$, we cannot have \[3=\Rank(\mathcal S\circ \mathcal T)
					\ >\ \Rank(\mathcal S)=2.\]

				\item The range of $\mathcal T$ is a plane, while the range of $\mathcal
					S$ is a line, and we would like $\mathcal S \circ \mathcal T$ to
					send every vector to $0$. Let $\mathcal S$ be projection onto the $xy$-plane
					and let $\mathcal T$ be projection onto the $z$-axis. Then
					$\Null(\mathcal S)=\Range(\mathcal T)$ and so $\mathcal S\circ\mathcal T=\mathbf 0$,
					the transformation which sends every vector to zero.
			\end{enumerate}
			\item Sorry Tommy, there must be a mistake somewhere in your lost notes.
				Comparing ranks, $\Rank(A)=3$ and $\Rank(B)=2$.
				Since $X$ is invertible, it does not change the
				dimension of the image of any subspace,
				so for similar matrices $A\sim B$, we must have $\Rank(A)=\Rank(B)$.

		\end{enumerate}