\subsection*{Learning Objectives}
	Students need to be able to\ldots
	\begin{itemize}
		\item write precise mathematical definitions of concepts used in class
		\item distinguish between a concept being ``familiar'' and actually knowing it
		\item determine whether vectors are in a span or not
		\item understand that you can take the span of linearly dependent sets (which
			might be a point of confusion when bases come around).
	\end{itemize}

\subsection*{Context}
	Students have spent several days on span and have gone over the definition of linearly independent
		and dependent. They have just started dot products and orthogonality.

\subsection*{What to Do}
	Introduce the learning objectives for the day's tutorial. Explain that linear algebra relies
		heavily on knowing precise definitions and even though you might ``feel'' like
		you know a concept, you don't really know it if you cannot write it down.
	
	Have students pair up and ask them to actually write down their definition for problem 1.
	Many will be tempted to ``do it in their head''. Don't let them!

	When many groups have something written, partition the board and invite several groups to
	write their definition on the board. Make sure to include a wrong definition (which will be
	most of them). Have the class read each definition and take a moment to vote
	``totally correct'', ``somewhat correct'', or `totally incorrect'' for each definition.
	Then, discuss what's wrong with each definition and how it might be fixed (if possible).
	\emph{Make it clear that on the midterms only ``totally correct'' definitions will get points;
	even a definition with 99\% of the correct words might get a zero if it is not totally correct.}

	After this discussion, have students move on to number 2. When most students have finished,
		have a quick discussion of number 2, rinse, and repeat.
	
	7 minutes before the end of tutorial, pick a problem most groups have started working on
	and do it as a wrap-up. Remember, the goal of tutorial is not to get through all the problems,
	and don't go over a problem the groups have not started yet---that won't be helpful to them.


\subsection*{Notes}
		\begin{enumerate}
			\item Most groups will have something wrong with their definition of span.
				Don't let them off the hook---make sure they understand that for
				definitions, there is no ``close''.
			\item Students wont know the difference between
				\[
			\{\vec x:\vec x=\alpha\vec u+\beta\vec v+\gamma \vec w\text{ for some }\alpha,\beta,\gamma\in \R\}
				\]
				and
				\[
			\{\vec x:\vec x=\alpha\vec u+\beta\vec v+\gamma \vec w\text{ for all }\alpha,\beta,\gamma\in \R\}.
				\]
				If this point comes up, emphasize it.
			\item If they're stuck on problem 2, remind them they can always use systems of linear equations to answer
				this question. They know how to solve \emph{any} system by now.

			\item If they're stuck on problem 3, ask them to write the plane down in vector form and then
				ask how the direction vectors might relate to the question.
		\end{enumerate}
