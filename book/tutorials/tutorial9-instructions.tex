	Students need to be able to\ldots
	\begin{itemize}
		\item Apply determinants to solve geometric questions
		\item Use linear algebra tools to answer open-ended questions (i.e., define ``pointiness'')
		\item Work with new definitions (cross product)
	\end{itemize}

\subsection*{Context}
	At this point, all students have seen determinants in class; additionally, they have had online homework
		where they had to compute $2\times 2$ and $3\times 3$ determinants.

		In this class, determinants are defined as the oriented volume of the image of the unit cube (for a
		transformation) and the oriented volume of the parallelepiped given by the column/row vectors
		(for a matrix). We do \emph{not} define the determinant in terms of cofactors, so don't expect students
		to know this or use it. However, they must know know to compute $2\times 2$ and $3\times 3$ determinants.

\subsection*{What to Do}
	Start the tutorial by stating the day's learning objectives. Re-emphasize that the goal
		of tutorial is not to get through many problems, but to spend quality time thinking
		and applying math tools we've learned.

		Have students start in groups on \#1. When most groups have answers for part (a),
		have a class discussion about how they found the volumes (this should be straightforward). Then,
		have groups resume working on part (b). Part (b) will be very hard. There isn't a ``right'' way
		to do this,
		and they'll need to be creative. After students have adequately struggled, you might choose
		to have a group discussion about ideas for (b). Some might come up with ideas involving volumes,
		others might want to use dot products to find angles. Get two different ideas on the board and then
		let students pick which one to implement.

		It is likely that \#1 takes the whole classtime. If you do have extra time, continue as usual,
		letting students work in small groups on \#2 and then having a mini-discussion when half the groups
		have figured it out.

		6 minutes before the end of class, pick a suitable problem to do as a wrap-up.

\subsection*{Notes}
	\begin{itemize}
		\item Students should know how to compute $2\times 2$ and $3\times 3$ determinants via
			an algorithm, but they will be slow at it. Let them take their time---today is their practice day!
		\item For \#1 the idea of ``pointiness'' isn't precise and there are several valid metrics.
			First, have them use pens/pencils as vectors and actually have them orient them the way the crystal is oriented.
			Can they judge based just on their ``3d diagram'' which is pointiest?

			After they have an estimate, try to get them to formalize their intuition.
			If they need a hint, have them think about angles, and
			how could they tell whether an angle is more acute or not.

		\item For \#3, the big barrier will be figuring out that $\vec f$ should be orthogonal
			to $\vec a$ and $\vec d$. Once they figure that out, they might be stuck on how to find
			such a vector. Remind them that they can set up a system of linear equations if they
			aren't able to guess a solution.

		\item \#4 would normally be a nightmare, but the problem is set up to make the numbers easy.
			They might not notice that $\vec u$ and $\vec v$ form the sides of a rectangle, so
			they won't need trigonometry to compute its area.

			The second stumbling block will be how to get a hold of a vector orthogonal to both
			$\vec u$ and $\vec v$. This one is hard to get by guessing. Have them write equations
			for what they need.
	\end{itemize}