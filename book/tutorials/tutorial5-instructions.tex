\subsection*{Learning Objectives}
	Students need to be able to\ldots
	\begin{itemize}
		\item Write vectors in multiple bases
		\item Write matrices for linear transformations in multiple bases
		\item Explain the pros and cons for one basis over another in a particular problem
	\end{itemize}

\subsection*{Context}
	In class, students have gone over bases, subspaces, dimension, linear transformations, and
		matrix transformations. In this class, we only work with $\R^n$ and subspaces of $\R^n$,
		so we won't be talking about bases for abstract spaces.
	
	Bases are especially hard for students---they're used to thinking of lists of numbers
		as the actual vectors instead of representations of the vectors. They need
		a lot of hand-holding to make this transition.

\subsection*{What to Do}
	Start by explaining to students that bases and linear transformations are \emph{the}
		two big ideas from linear algebra and that today we're going to focus on bases.
		Further tell them that up till now we've been considering vectors and lists
		of numbers as interchangeable, but now we think of lists of numbers
		as a \emph{representation} of a vector instead of a true vector (kind of like
		the symbol ``4'' is not literally four, it is a graphical representation of
		the abstract idea of the number four).
	
	Proceed as usual, asking students to form small groups and start working.
		Again, this tutorial starts with a definition question, which they all need to actually write out.
		The meat of this tutorial is problem \#2.
		When most groups are on \#2(c), have a class discussion on 2(a) and 2(b), to
		make sure everyone is on the same page.


	Remember, the point of this tutorial (and all tutorials) is not to get through all the problems.
	It is to get practice with difficult and new mathematics. Don't cut thinking time short trying
	to get through more problems!

	8 minutes before the end of class, pick a problem that most groups have started, or a problem
		that they've finished but you think needs more attention, and do that problem as a wrap-up.


\subsection*{Notes}
		\begin{enumerate}
			\item To talk about representations in a basis we technically need \emph{ordered} sets. In this class,
				when writing a set down and declaring it a basis, we imply that the order of the basis vectors
				is the left-to-right order that they're written in. The student's won't think about regular sets lacking
				order,
				so don't bring it up unless they ask.
			\item We use the notation $[\vec v]_{\mathcal X}$ to mean the list of numbers representing
				$\vec v$ in the basis $\mathcal X=\{\vec x_1,\vec x_2,\ldots\}$. We use $\mat{1\\2\\\vdots}_{\mathcal X}$ to
				mean the linear combination $1\vec x_1+2\vec x_2+\cdots$. Thus $[[\vec v]_{\mathcal X}]_{\mathcal X}=\vec v$.
			\item Lists of number that are not subscripted and are treated as vectors have an implicit
				subscript of $\mathcal E$, the standard basis. Students may be uncomfortable
				that sometimes we demand subscripts and sometimes we don't. It should always be clear from context
				whether we are talking about a list of numbers or a vector written in the standard basis, and on
				tests we will be very clear.
			\item For 2(c), make sure that the shortcut comes out that $[7\vec v]_{\mathcal X} = 7[\vec v]_{\mathcal X}$ and,
				in fact, changing basis is a linear operation.
			\item Problems 2(d) and 4 ask for value judgments. There's no ``right'' answer to these questions, but
				some answers are better reasoned than others. Hold the students accountable for coming
				up with good reasons.
			\item No one will get to \#4, but if they do, encourage them to use a calculator to speed things along.
		\end{enumerate}