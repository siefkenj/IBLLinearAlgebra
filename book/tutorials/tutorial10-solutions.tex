		\begin{enumerate}
			\item \begin{enumerate}
					\item $\vec v$ is an eigenvector for $T$ if $\vec v\neq \vec 0$ and
						$T\vec v=\lambda \vec v$ for some scalar $\lambda$.
					\item $\vec v$ is an eigenvector for $T$ if it doesn't change directions
						when $T$ is applied.
			\end{enumerate}
			\item
				For $A$:
				$\vec v_1$ is an eigenvector with eigenvalue $-1$;
				$\vec v_2$ is an eigenvector with eigenvalue $2$.

				For $B$:
				$\vec v_2$ is an eigenvector with eigenvalue $2$;
				$\vec v_4$ is an eigenvector with eigenvalue $-1$.

				For $C$:
				$\vec v_4$ is an eigenvector with eigenvalue $0$.
			\item $\mathcal P$ has eigenvalues of $0$ and $1$. All non-zero multiples of $\mat{0\\1}$
				have eigenvalue $0$ and non-zero multiples of $\mat{1\\0}$ have eigenvalue $1$.

				$\mathcal R$ has no real eigenvalues and no real eigenvectors. Every non-zero
				vector changes direction when $\mathcal R$ is applied.
			\item It is not possible.

				\emph{Proof:} Since $\vec v$ is an eigenvector for $T$ with eigenvalue
				$2$, we know $T\vec v=2\vec v$. Since $T$ is linear, we know
				\[
					T(7\vec v)=7T\vec v=14\vec v,
				\]
				and so $7\vec v$ is an eigenvector for $T$ with eigenvalue $2$. There is no
				other possibility.
			\item \begin{enumerate}
				\item $T^{100}\vec w=\vec v_1+\tfrac{1}{2^{100}}\vec v_2 \approx \vec v_1$.
				\item Yes. Since $\vec v_1$ and $\vec v_2$ are eigenvectors, neither is zero.
				Suppose $\{\vec v_1,\vec v_2\}$ is linearly dependent. Then, $\vec v_1=t\vec v_2$
					for some $t$ (because $\vec v_2\neq \vec 0$). Computing,
					\[
						T\vec v_1=T(t\vec v_2)=tT\vec v_2=\tfrac{t}{2}\vec v_2 = \tfrac{1}{2}\vec v_1,
					\]
					and so $\vec v_1$ would have an eigenvalue of $\tfrac{1}{2}$. But, this
					is a contradiction since the eigenvalue
					of $\vec v_1$ is $1$. Thus $\{\vec v_1,\vec v_2\}$ must be linearly independent.

					Finally, a linearly independent set of two vectors in $\R^2$ must span all of $\R^2$,
					and so $\{\vec v_1,\vec v_2\}$ is a basis for $\R^2$.
				\item Yes. First write $\mat{a\\b}$ in the $\{\vec v_1,\vec v_2\}$ basis. Then throw away
					the $\vec v_2$ component.

					The matrix that converts from the standard basis to the $\{\vec v_1,\vec v_2\}$ basis is
					\[
						C=\mat{-5&3\\2&-1}.
					\]
					The matrix that drops second coordinate of a vector is
					\[
						D=\mat{1&0\\0&0}.
					\]
					Thus
					\[
						T^{100}\mat{a\\b} \approx C^{-1}DC\mat{a\\b}=\mat{-5a+3b\\-10a+6b}.
					\]
				\item Yes. Suppose $S$ has a unique, largest, positive eigenvalue $\lambda$. Then $S'=\tfrac{1}{\lambda}S$
					behaves similarly to $T$. Approximate the same way, then multiply by $\lambda$.
			\end{enumerate}
		\end{enumerate}