\documentclass[11pt]{exam}
\usepackage{amsmath}
\usepackage{amssymb}
\usepackage{graphicx}
\usepackage{enumitem}
\usepackage{amsfonts}
\usepackage{amssymb}
\usepackage{xparse}
\usepackage{ifthen}
\usepackage{geometry}
\noprintanswers

\newcommand {\DS} [1] {${\displaystyle #1}$}
\newcommand{\answer}[1]{{\bf Answer:} \; #1}
\newcommand{\vv}{\vspace{.2cm}}
\newcommand{\vvv}{\vspace{6cm}}

\newcommand{\ul}{$\underline{\phantom{xxx}}$}
\newcommand{\ull}{\underline{\phantom{xxx}}}
\newcommand{\xh}{\hat{\bf x}}
\newcommand{\yh}{\hat{\bf y}}
\newcommand{\zh}{\hat{\bf z}}
\newcommand{\R}{\mathbb{R}}
\newcommand{\C}{\mathbb{C}}
\newcommand{\Z}{\mathbb{Z}}
\newcommand{\N}{\mathbb{N}}
\newcommand{\proj}{\mathrm{proj}}
\newcommand{\mat}[1]{\begin{bmatrix}#1\end{bmatrix}}
\newcommand{\floor}[1]{\lfloor #1 \rfloor}

\renewcommand{\span}{\mathrm{span}\,}
\newcommand{\rref}{\mathrm{rref}}
\newcommand{\rank}{\mathrm{rank}}
\newcommand{\nnul}{\mathrm{nullity}}

\pagestyle{empty}


%%%%%%%%%%%%%%%%%%%%%%%%%%%%%%%%%%%%%%%%%
%  Edit course information here
%%%%%%%%%%%%%%%%%%%%%%%%%%%%%%%%%%%%%%%%%

\newcommand{\mthCourse}{MATH 110}
\newcommand{\mthTerm}{Fall 2013}
\newcommand{\mthTutorialNumber}{8}
\newcommand{\mthDate}{October 30, 2013}


%%%%%%%%%%%%%%%%%%%%%%%%%%%%%%%%%%%%%%%%%
\topmargin -1in
\textheight 10in

\begin{document}


%%%%%%%%%%%%%%%%%%%%%%%%%%%%%%%%%%%%%%%%%%%%%%%
% Main Questions
%%%%%%%%%%%%%%%%%%%%%%%%%%%%%%%%%%%%%%%%%%%%%%%
{\large
	\begin{center}
		{\bf \mthCourse, \mthTerm}\\ 
		{\bf Tutorial \#\mthTutorialNumber}\\
		{\bf \mthDate}
	\end{center}
}

\section*{Today's main problems}

\begin{enumerate}
	\item 
		The transformation $\mathcal S:\R^2\to\R^2$ is
		given by $\mathcal S\mat{x\\y}=\mat{2x\\y}$ is linear.
	\begin{enumerate}
		\item Find the standard matrix representation for the transformation $\mathcal S$.
		\item Explain why this linear transformation is called 
			a \emph{horizontal scaling}.  Find the factor of this scaling.
			Is this transformation stretching or contracting vectors in $\R^2$?
		\item Describe in words what $\mathcal S^{-1}$ does to vectors in $\R^2$.
	\end{enumerate}
	
	\item Give an example of a \emph{vertical scaling} that contracts
	vectors in $\R^3$ by a factor of 3.

	\item Let $T$ denote the triangle with vertices give by the vectors
		\[
			\vec v_1=\mat{1\\1}\qquad
			\vec v_2=\mat{3\\1}\qquad
			\vec v_3=\mat{2\\3}
		\]
	\begin{enumerate}
		\item Draw the triangle $T$.
		\item Draw the result of $\mathcal S (T)$.
		\item $\mathcal C$ contracts vertically by a factor of $3$.
			Draw $\mathcal C(T)$.
		\item Draw $\mathcal S\mathcal C(T)$.  How does this compare
			to $\mathcal C\mathcal S(T)$?  Using this fact,
			what can you say about the
			matrix representations of $\mathcal S$ and $\mathcal C$?
	\end{enumerate}

\end{enumerate}
\subsection*{Further Questions}
\begin{enumerate}[resume]
	\item $\mathcal R:\R^2\to\R^2$ rotates vectors counterclockwise
		by $90^{\circ}$.
		\begin{enumerate}
			\item Compute the matrix representation of $\mathcal R$
				and $\mathcal R^{-1}$.
			\item Compute the matrix representation of the transformation $\mathcal W$
				that rotates vectors in $\R^2$ clockwise by $90^\circ$.
			\item How do $\mathcal R^{-1}$ and $\mathcal W$ relate?
		\end{enumerate}
	\item Consider the matrix $P$ that projects vectors in $\R^2$ onto
		the line $y=2x$.
		\begin{enumerate}
			\item Draw the image of $P$ applied to the triangle $T$ from
				problem 3.
			\item If you write down the matrix for $P$, would it be invertible?
				Why or why not?
		\end{enumerate}
\end{enumerate}




%%%%%%%%%%%%%%%%%%%%%%%%%%%%%%%%%%%%%%%%%%%%%%%
% Challenge questions
%%%%%%%%%%%%%%%%%%%%%%%%%%%%%%%%%%%%%%%%%%%%%%%
\newpage
{
	\begin{center}
		{\bf \mthCourse, \mthTerm}\\ 
		{\bf Tutorial \#\mthTutorialNumber}\\
		{\bf \mthDate}
	\end{center}
}

\section*{Challenge questions}

	$\mathcal R_t$ rotates vectors in $\R^2$ counterclockwise
		by $t$ radians.
\begin{enumerate}[resume]
	\item   Write the matrix representation, $R$, of $\mathcal R_t$.
	\item Write the matrix representation of $\mathcal R_t^{-1}$. (It's
		easier to think about this one instead of row reducing).
	\item How does $\mathcal R_t^{-1}$ compare to $R^T$?.
	\item Suppose $V$ is an $n\times n$ matrix whose columns are orthogonal
		and are all unit vectors.  Explain how $V^T$ relates to $V^{-1}$.
		Can you relate this to the rotation matrices you've been playing
		around with?

\end{enumerate}



%%%%%%%%%%%%%%%%%%%%%%%%%%%%%%%%%%%%%%%%%%%%%%%
% TA instructions
%%%%%%%%%%%%%%%%%%%%%%%%%%%%%%%%%%%%%%%%%%%%%%%
\newpage
{\small
	\begin{center}
		{\bf \mthCourse, \mthTerm}\\ 
		{\bf Tutorial \#\mthTutorialNumber. Instructions for TAs}
	\end{center}
}

\subsection*{Objectives}


\subsection*{Hidden objectives}


\subsection*{Suggestions}
	When they start scaling the triangle in number 3, make sure they scale from the origin 
	not just draw wider and shorter triangles (like they will be tempted to).

\subsection*{Wrapup}
	Choose a question that most of the class has started but not yet finished,
	or a question that people particularly struggled with.

\subsection*{Solutions}
\begin{enumerate}
	\item
\end{enumerate}
	

\end{document}
