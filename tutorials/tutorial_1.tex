\documentclass[11pt]{exam}
\usepackage{amsmath}
\usepackage{amssymb}
\usepackage{graphicx}
\usepackage{enumitem}
\usepackage{amsfonts}
\usepackage{amssymb}
\usepackage{xparse}
\usepackage{ifthen}
\usepackage{geometry}
\noprintanswers

\newcommand {\DS} [1] {${\displaystyle #1}$}
\newcommand{\answer}[1]{{\bf Answer:} \; #1}
\newcommand{\vv}{\vspace{.2cm}}
\newcommand{\vvv}{\vspace{6cm}}

\newcommand{\ul}{$\underline{\phantom{xxx}}$}
\newcommand{\ull}{\underline{\phantom{xxx}}}
\newcommand{\xh}{\hat{\bf x}}
\newcommand{\yh}{\hat{\bf y}}
\newcommand{\zh}{\hat{\bf z}}
\newcommand{\R}{\mathbb{R}}
\newcommand{\C}{\mathbb{C}}
\newcommand{\Z}{\mathbb{Z}}
\newcommand{\N}{\mathbb{N}}
\newcommand{\proj}{\mathrm{proj}}
\newcommand{\mat}[1]{\begin{bmatrix}#1\end{bmatrix}}
\newcommand{\floor}[1]{\lfloor #1 \rfloor}

\pagestyle{empty}


%%%%%%%%%%%%%%%%%%%%%%%%%%%%%%%%%%%%%%%%%
%  Edit course information here
%%%%%%%%%%%%%%%%%%%%%%%%%%%%%%%%%%%%%%%%%

\newcommand{\mthCourse}{MATH 110}
\newcommand{\mthTerm}{Fall 2013}
\newcommand{\mthTutorialNumber}{1}
\newcommand{\mthDate}{September 11, 2013}


%%%%%%%%%%%%%%%%%%%%%%%%%%%%%%%%%%%%%%%%%


\begin{document}


%%%%%%%%%%%%%%%%%%%%%%%%%%%%%%%%%%%%%%%%%%%%%%%
% Main Questions
%%%%%%%%%%%%%%%%%%%%%%%%%%%%%%%%%%%%%%%%%%%%%%%
{\large
	\begin{center}
		{\bf \mthCourse, \mthTerm}\\ 
		{\bf Tutorial \#\mthTutorialNumber}\\
		{\bf \mthDate}
	\end{center}
}

\section*{Today's main problems}
\begin{enumerate}
	\item Knowing the definitions is half the battle in Linear Algebra.
	Write the definitions of the following (either in words or with a formula).
	\begin{itemize}
		\item  \emph{Vector}: \vv\vv\vv
		\item  \emph{Dot product}: \vv\vv\vv
		\item  The \emph{norm} of $\vec w$: \vv\vv\vv
		\item  \emph{Distance between} vectors $\vec u$ and $\vec v$: \vv\vv\vv
		\item  $\vec u$ and $\vec v$ are \emph{orthogonal} when: \vv\vv\vv
		\item  $\vec w$ is a \emph{linear combination} of $\vec u$ and $\vec v$ when: \vv\vv\vv
		\item  $\vec w$ is a \emph{unit vector} when: \vv\vv\vv
	\end{itemize}

	\item For the vectors $\vec u=\mat{-2\\1\\-2}$ and $\vec v=\mat{4\\-1\\3}$ determine
	\begin{enumerate}
		\item the distance between $\vec u$ and $\vec v$
		\item a unit vector in the direction $\vec v$
		\item whether $\vec u$ and $\vec v$ are orthogonal
		\item the angle between $\vec u$ and $\vec v$
	\end{enumerate}
\end{enumerate}


\section*{Further questions}

\begin{enumerate}[resume]
	\item  Find the projection of $\vec u$ onto $\vec v$.

	\item $L=\{t\vec v: t\in \R\}$ is the line in the direction $\vec v$ that
	passes through the origin.  Find the intersection of $L$ with the sphere
	of radius 2 centered at the origin.
\end{enumerate}

%%%%%%%%%%%%%%%%%%%%%%%%%%%%%%%%%%%%%%%%%%%%%%%
% Challenge questions
%%%%%%%%%%%%%%%%%%%%%%%%%%%%%%%%%%%%%%%%%%%%%%%
\newpage
{
	\begin{center}
		{\bf \mthCourse, \mthTerm}\\ 
		{\bf Tutorial \#\mthTutorialNumber}\\
		{\bf \mthDate}
	\end{center}
}

\section*{Challenge questions}

\begin{enumerate}[resume]

	\item  Find a vector that forms a $45^\circ$ angle with $\vec v$ and has length 7. 

	\emph{Hint:} There are many, many such vectors, so you'll have to make some choices.

	\item The vector $\vec w$ forms an angle of $45^\circ$ with $\vec v$ and
	$\|\proj_{\vec v}\vec w\|=10$.  The vector $\vec r$ satisfies
	$\|\vec r\| = 2\|\vec w\|$ and $\|\proj_{\vec v}\vec r\|=10$.  What are the
	possible angle(s) between the vectors $\vec r$ and $\vec v$?

\end{enumerate}



%%%%%%%%%%%%%%%%%%%%%%%%%%%%%%%%%%%%%%%%%%%%%%%
% TA instructions
%%%%%%%%%%%%%%%%%%%%%%%%%%%%%%%%%%%%%%%%%%%%%%%
\newpage
{\small
	\begin{center}
		{\bf \mthCourse, \mthTerm}\\ 
		{\bf Tutorial \#\mthTutorialNumber. Instructions for TAs}
	\end{center}
}

\subsection*{Objectives}

	Knowing the definitions is half the battle in Linear Algebra.  Today
	we are going to focus on knowing the definitions and how to apply them.

\subsection*{Hidden objectives}
	
	Everyone forgets a definition at some point, but the key is to be able to
	find it when you need it.  We'd like students to be resourceful and use
	each other and their \emph{textbook} to find the definitions they don't know.

\subsection*{Suggestions}
	Ask students to work on number 1 first and to talk to their neighbors 
	and refer to their textbook and notes if they are unsure.  Give them
	sufficient time come up with definitions before going over them 
	midway through the class.  Going over them doesn't need to take much time
	since they have seen these definitions before, but we want everyone to be
	on the same page before starting 2.

	Further, since this is the first tutorial, make sure to explain that there
	are many questions and they are not expected to complete them all during
	tutorial time.

\subsection*{Wrapup}
	Choose a question that most of the class has started but not yet finished,
	or a question that people particularly struggled with.

\subsection*{Solutions}
\begin{enumerate}
	\item Knowing the definitions is half the battle in Linear Algebra.
	Write the definitions of the following (either in words or with a formula).
	\begin{itemize}
		\item  \emph{Vector}: A magnitude and a direction; a list of components.
		\item  \emph{Dot product}: $\vec a\cdot \vec b=\|\vec a\|\|\vec b\|\cos \theta$
		where $\theta$ is the angle between them; $\vec a\cdot \vec b=\sum a_ib_i$
		where $a_i$ are the components of $\vec a$ and $b_i$ are the components 
		of $\vec b$.
		\item  The \emph{norm} of $\vec w$: the length of $\vec w$; $\|\vec w\|=\sqrt{
		\vec w\cdot \vec w}$ 
		\item  \emph{Distance between} vectors $\vec u$ and $\vec v$: $\|\vec u-\vec v\|$
		\item  $\vec u$ and $\vec v$ are \emph{orthogonal} when: $\vec u$ and $\vec v$
		are perpendicular; $\vec u$ and $\vec v$ meet at $90^\circ$; $\vec u\cdot \vec v=0$
		\item  $\vec w$ is a \emph{linear combination} of $\vec u$ and $\vec v$ when: 
		$\vec w=a\vec u+b\vec v$ for some numbers $a,b$
		\item  $\vec w$ is a \emph{unit vector} when: $\|\vec w\|=1$
	\end{itemize}

	\item For the vectors $\vec u=\mat{-2\\1\\-2}$ and $\vec v=\mat{4\\-1\\3}$ determine
	\begin{enumerate}
		\item the distance between $\vec u$ and $\vec v$

		$\sqrt{65}\approx 8.06$
		\item a unit vector in the direction $\vec v$
		
		$\frac{1}{\sqrt{26}}\vec v$
		\item whether $\vec u$ and $\vec v$ are orthogonal

		No
		\item the angle between $\vec u$ and $\vec v$

		$\cos \theta = \frac{-15}{3\sqrt{26}}\approx-0.98$ so $\theta \approx 169^\circ$
	\end{enumerate}
	\item  Find the projection of $\vec u$ onto $\vec v$.

	$\frac{1}{26}\mat{-60\\15\\-45}\approx\mat{-2.31\\.58\\-1.73}$
	\item $L=\{t\vec v: t\in \R\}$ is the line in the direction $\vec v$ that
	passes through the origin.  Find the intersection of $L$ with the sphere
	of radius 2 centered at the origin.

	Two points: $\frac{2}{\sqrt{26}}\mat{4\\-1\\3}$ and $\frac{-2}{\sqrt{26}}\mat{4\\-1\\3}$
\end{enumerate}
	

\end{document}
