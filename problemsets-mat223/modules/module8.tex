
Recall that when we write $\vec x=\mat{2\\3}$, what we actually mean is $\vec x=2\xhat+3\yhat$. The numbers $2$
and $3$ are called the coordinates of the vector $\vec x$ with respect to the standard basis.  However, in general,
subspaces have many bases, and so it is possible to represent a single vector in many ways as coordinates with
respect to \emph{many} bases.

Let $\vec x=\mat{2\\3}$; let $\mathcal E=\Set{\xhat,\yhat}$ be the standard basis for $\R^2$ and let
$\mathcal B=\Set{\vec b_1,\vec b_2}$ where $\vec b_1=\mat{1\\1}$ and $\vec b_2=\mat{1\\2}$ be another
basis for $\R^2$. Now, the coordinates of $\vec x$ with respect to $\mathcal E$ are $(2,3)$, but
the coordinates of $\vec x$ with respect to $\mathcal B$ are $(1,1)$.

XXX Figure

The coordinates $(2,3)$ and $(1,1)$ represent $\vec x$ equally well, and when solving problems, we should pick the
coordinates that make our problem the easiest\footnote{ For example, maybe in one choice of coordinates, we can avoid all 
fractions in our calculations---this could be good if you're programming a computer that rounds it.}. However, now that we
are representing vectors in multiple bases, we need a way to keep track of what coordinates correspond to which basis.

\SavedDefinitionRender{RepresentationinaBasis}

\begin{example}
	Let $\mathcal E=\Set{\xhat,\yhat}$ be the standard basis for $\R^2$ and let $\mathcal B=\Set{\vec b_1,\vec b_2}$
	where $\vec b_1=\xhat+\yhat$, and $\vec b_2 =3\yhat$ be another basis for $\R^2$. Given that $\vec v=2\xhat-\yhat$, 
	find $[\vec v]_{\mathcal E}$ and $[\vec v]_{\mathcal B}$.

	XXX Finish
\end{example}

\Heading{Notation Conventions}
In light of this notation, we need to revisit some past notation. Again, we have been writing $\vec x=\mat{1\\3}$ to mean
$\vec x=2\xhat+3\yhat$. However, given the representation-in-a-basis notation, we should be writing
\[
	\vec x=\mat{2\\3}_{\mathcal E},
\]
where $\mathcal E$ is the standard basis for $\R^2$. We should write $\mat{2\\3}_{\mathcal E}$ because the coordinates $(2,3)$
refer to \emph{different} vectors for \emph{different} bases. However, most of the time we are only thinking about the standard
basis. So, the convention we will follow is:
\begin{itemize}
	\item If a problem involves only one basis, we may write $\mat{x\\y}$ to mean $\mat{x\\y}_{\mathcal E}$ where
	$\mathcal E$ is the standard basis.
	\item If there are multiple bases in a problem, we will always write $\mat{x\\y}_{\mathcal X}$ to specify a vector in
	coordinates relative to a particular basis $\mathcal X$.
\end{itemize}

\begin{emphbox}[Takeaway]
	If a problem only involves the standard basis, we may use the notation we always have. If a problem involves
	multiple bases, we must \emph{always} use representation-in-a-basis notation.
\end{emphbox}


\Heading{True Vectors vs\mbox{.} Representations}

\begin{center}
\includegraphics[width=3in]{images/MagrittePipe.jpg}\footnote{Image take from Wikipedia: \url{https://en.wikipedia.org/wiki/File:MagrittePipe.jpg}} 
\end{center}
The Belgian surrealist Ren\'e Magritte painted the work above, which is subtitled, ``This is not a pipe''. Why? Because, of course, it is not
a pipe. It is a painting of a pipe! In this work, Magritte points out a distinction that will soon become very important to
us---the distinction between an object and a representation of that object.



Let $\vec x=2\xhat+3\yhat\in \R^2$. The vector $\vec x$ is a \emph{real-life geometrical thing}, and to
emphasize this, we will call $\vec x$ a \emph{true} vector. In contrast, when we write
the column matrix $[\vec x]_{\mathcal E}=\mat{2\\3}$, we are writing a \emph{list of numbers}. The list of
numbers $\mat{2\\3}$ has no meaning until we give it a meaning by assigning it a basis. For example,
by writing $\mat{2\\3}_{\mathcal E}$ we declare that the numbers $2$ and $3$ are coefficients of $\xhat$ and
$\yhat$. By writing $\mat{2\\3}_{\mathcal B}$ where $\mathcal B=\Set{\vec b_1,\vec b_2}$,
we declare that the numbers $2$ and $3$ are coefficients of $\vec b_1$ and $\vec b_2$. Since a list of numbers
without a basis has no meaning, we must write
\[
	\vec x\,\,{\color{Red}\neq}\,\,\, [\vec x]_{\mathcal E}=\mat{2\\3},
\]
since the left side of the equation is a \emph{true vector} and the right side is a \emph{list of numbers}. Similarly,
we must write
\[
	[\vec x]_{\mathcal E}\,\,{\color{Red}\neq}\,\,\, \mat{2\\3}_{\mathcal E}=\vec x,
\]
since the left side is a \emph{list of numbers} and the right side is a \emph{true vector}.

To help keep the notation straight in your head, for a basis $\mathcal X$, remember the rule
\[
	[\text{true vector}]_{\mathcal X} = \text{list of numbers}\qquad\text{and}\qquad
	[\text{list of numbers}]_{\mathcal X} =\text{true vector}.
\]

It's easy to get confused when answering questions that involve multiple bases; precision will
make these problems much easier.

\Heading{Orientation of a Basis}


