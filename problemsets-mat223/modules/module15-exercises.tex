
\begin{exercises}

	\begin{problist}
		\prob For each linear transformation defined below, find its eigenvectors
		and eigenvalues. If it has no eigenvectors/values, explain why not.
		\begin{enumerate}
			\item $\mathcal S:\R^{2}\to\R^{2}$, where $\mathcal S$ stretches
				every vector by the factor of $3$.

			\item $\mathcal R:\R^{2}\to\R^{2}$, where $\mathcal R$ rotates
				every vector clockwise by $\frac{\pi}{4}$.

			\item $\mathcal P:\R^{2}\to\R^{2}$, where $\mathcal P$ projects
				every vector onto the line $\ell$ given by $y=-x$.

			\item $\mathcal F:\R^{2}\to\R^{2}$, where $\mathcal F$ reflects
				every vector over the line $\ell$ given by $y=-x$.

			\item $T:\R^{3}\to\R^{3}$, where $T$ is a linear transformation
				induced by the matrix $\mat{1&2&3\\3&4&5\\5&6&7}$.

			\item $U:\R^{3}\to\R^{2}$, where $U$ is a linear transformation
				induced by the matrix $\mat{1&2&3\\3&4&5}$.
		\end{enumerate}

		% Q1 Solutions

		\begin{solution}

			\begin{enumerate}
				\item Every non-zero vector in $\R^{3}$ is an eigenvector with
					eigenvalue 3.

				\item $\Char(R)$ has no real root, so $R$ has no real eigenvalue
					or eigenvectors.

				\item There are two eigenvalues. $0$ is an eigenvalue with
					eigenvector $\mat{1 \\ 1}$, and $1$ is an
					eigenvalue with eigenvector $\mat{1\\-1}$.

				\item There are two eigenvalues. $-1$ is an eigenvalue with
					eigenvector $\mat{1\\1}$, and $1$ is an eigenvalue
					with eigenvector $\mat{1 \\ -1}$.

				\item $\Char(T) = - \lambda ( \lambda^{2} - 12 \lambda - 12)$.
					Then, we have three eigevalues. $0$ is an
					eigenvalue with eigenvector $\mat{1 \\ -2 \\ 1}$, $6
					+ 4 \sqrt{3}$ is an eigenvalue with eigenvector $\mat{2
					\\ 2 + \sqrt{3} \\ 0}$, and $6 - 4\sqrt{3}$ is an
					eigenvalue with eigenvector
					$\mat{2 \\ 2 - \sqrt{3} \\ 0}$.

				\item $U$ is induced by a $2 \times 3$ matrix, and eigenvalues/eigenvectors
					are only defined for linear maps from $\R^{n}$ to itself.
					So, $U$ has no eigenvalues or eigenvectors.
			\end{enumerate}
		\end{solution}

		\prob Let $A = \mat{a&b\\c&d}$, where $a,b,c,d \in \R$.
		\begin{enumerate}
			\item Find the characteristic polynomial of $A$.

			\item Find conditions on $a,b,c,d$ so that $A$ has (i) two distinct
				real eigenvalues, (ii) exactly one real eigenvalue, (iii) no
				real eigenvalues.
		\end{enumerate}

		%Q2 Solutions

		\begin{solution}

			\begin{enumerate}
				\item By definition,
					\begin{align*}
						\Char(A) = \det (A - \lambda I) = \det \mat{a - \lambda & b \\ c & d - \lambda}= \lambda^{2} - (a + d)\lambda + ad - bc.
					\end{align*}

				\item By the quadratic formula, the discriminant $\Delta$ of
					$\Char(A)$ is $\Delta = (a + d)^{2} - 4(ad - bc) =
					(a - d)^{2} + 4 bc$. So, $A$ has two distinct real
					eigenvalues if $(a - d)^{2} + 4bc > 0$, one real eigenvalue
					if $(a - d)^{2} + 4bc = 0$, and no real
					eigenvalues if $(a - d)^{2} + 4bc < 0.$
			\end{enumerate}
		\end{solution}

		\prob Let $B=\mat{1&2\\0&4}$.
		\begin{enumerate}
			\item Find the eigenvalues of $B$.

			\item Find the eigenvalues of $B^{T}$.

			\item A vector $\vec v\neq\vec 0$ is called a \emph{left-eigenvector}
				for $B$ if $\vec vB=\lambda \vec v$ for some scalar
				$\lambda$ (Here we consider $\vec v$ a \emph{row} vector).
				Find all left eigenvectors for $B$.
		\end{enumerate}

		% Q3 Solutions

		\begin{solution}

			\begin{enumerate}
				\item $\Char(B) = \det\mat{1 -\lambda & 2 \\ 0 & 4 - \lambda}=
					(1 - \lambda)(4 - \lambda)$. So, $B$ has
					eigenvalues $1$ and $4$.

				\item $\Char(B) = \Char(B^{T})$, so $B^{T}$ also has
					eigenvalues $1$ and $4$.

				\item $\vec{v}^{T}B = \lambda \vec{v}^{T}$ if and only if $B^{T}
					\vec{v}= \lambda \vec{v}$ where we use $\vec{v}$ for
					a column vector and $\vec{v}^{T}$ for its corresponding
					row vector. We observe that $B^{T}$ has eigenvector
					$\mat{1 \\ 0}$ with eigenvalue $1$ and eigenvector
					$\mat{0\\1}$ with eigenvalue $4$. Hence $B$ has
					left eigenvector $\mat{1 & 0}$ with eigenvalue $1$
					and left eigenvector $\mat{0 & 1}$ with eigenvalue
					$4$.$\vec{v}^{T}B = \lambda \vec{v}^{T}$ if and
					only if $B^{T} \vec{v}= \lambda \vec{v}$ where we
					use $\vec{v}$ for a column vector and
					$\vec{v}^{T}$ for its corresponding row vector. We
					observe that $B^{T}$ has eigenvector
					$\mat{1 \\ 0}$ with eigenvalue $1$ and eigenvector
					$\mat{0\\1}$ with eigenvalue $4$. Hence $B$ has left
					eigenvector $\mat{1 & 0}$ with eigenvalue $1$ and left
					eigenvector $\mat{0 & 1}$ with eigenvalue $4$.
			\end{enumerate}
		\end{solution}

		\prob For each statement below, determine whether it is true or false. Justify
		you answer.
		\begin{enumerate}
			\item Zero cannot be an eigenvalue of any matrix.

			\item $\vec 0$ cannot be an eigenvector of any matrix.

			\item A $2\times 2$ matrix always has a real eigenvalue.

			\item A $3\times 3$ matrix always has a real eigenvalue.

			\item A $3\times 2$ matrix always has a real eigenvalue.

			\item The matrix $M = \mat{3&3&3&3\\3&3&3&3\\3&3&3&3\\3&3&3&3}$ has
				exactly one eigenvalue.

			\item An invertible matrix can never have zero as an eigenvalue.
		\end{enumerate}

		% Q4 Solutions

		\begin{solution}

			\begin{enumerate}
				\item False. $0$ is an eigenvalue of $\mat{0 & 0 \\ 0 & 0}$.

				\item True. An eigenvector is a nonzero vector by definition.

				\item False. $\mat{0 & 1 \\ -1 & 0}$ has no real eigenvalue.

				\item True. It's characteristic polynomial has degree $3$ and
					hence has at least one real root.

				\item False. Eigenvalues are not defined for a non-square matrix.

				\item False. $\mat{3 & 3 & 3 & 3 \\ 3 & 3 & 3 & 3\\3 & 3 &
					3 & 3\\3 & 3 & 3 & 3}$ has $0$ as an eigenvalue with
					eigenvector $\mat{1 \\ -1 \\ 0 \\ 0}$. It also has
					eigenvalue $12$ with eigenvector
					$\mat{1 \\ 1 \\ 1 \\ 1}$.

				\item True. Any eigenvector with eigenvalue $0$ lies in the
					null space of the matrix, which implies that the null
					space has dimension at least one.
			\end{enumerate}
		\end{solution}
	\end{problist}
\end{exercises}