\begin{exercises}
	\begin{problist}
		\prob  Express the following lines in vector form.
		\begin{enumerate}
			\item   $\ell_1\subseteq\R^2$ with equation $4x-3y=-10$. 
			\item   $\ell_2\subseteq\R^2$ which passes through the points $A=(1,1)$ and $B=(2,7)$.
			\item   $\ell_3\subseteq\R^2$ which passes through $\vec 0$ and is parallel to the line
				with equation $4x-3y=-10$.
			\item   $\ell_4\subseteq\R^3$ which passes through the points $A=(-1,-1,0)$ and $B=(2,3,5)$.
			\item   $\ell_5\subseteq\R^3$ which is contained in the $yz$-plane and where the coordinates
				of every point in $\ell_5$ satisfy $x+2y-3z=5$.
		\end{enumerate}
		% Q1 Solutions
		\begin{solution}
    		\begin{enumerate}
    		    \item $\vec v = s\mat{3\\4} + \mat{-4\\-2} $
    		    \item $\vec v = s\mat{1\\6} + \mat{1\\1} $
    		    \item $\vec v = s\mat{3\\4}$
    		    \item $\vec v = s\mat{3\\4\\5} + \mat{-1\\-1 \\0}$
    		    \item $\vec v = s\mat{0\\3\\2} + \mat{0\\1\\-1}$
    		\end{enumerate}
		\end{solution}
		\prob Express the following planes in vector form
		\begin{enumerate}
			\item   $\mathcal P_1\subseteq\R^3$ with equation $4x-z=0$.
			\item   $\mathcal P_2\subseteq\R^3$ which passes through the points $A=(-1,-1,0)$, $B=(2,3,5)$, and $C=(3,3,3)$.
			\item   $\mathcal P_3\subseteq\R^3$ with equation $4x-3y+z=-10$.
			\item   $\mathcal P_4\subseteq\R^3$ which is parallel to the $yz$-plane but passes through the point $X=(1,-1,1)$.
			\item   $\R^2$.
			\item   $\mathcal P_5\subseteq\R^4$ which passes through $A=(1,-1,1,-1)$,
				and where the coordinates of every point in $\mathcal P_5$ satisfy the equations $x+y+2z-w=3$
				and $x+y+z+w=0$.
		\end{enumerate}
		% Q2 Solutions
		\begin{solution}
    		\begin{enumerate}
    		    \item $\vec v = s\mat{0\\1\\0} + t\mat{1\\0\\4}$
    		    \item $\vec v = s\mat{3\\4\\5} + t\mat{4\\4\\3} + \mat{-1\\-1\\0}$
    		    \item $\vec v = s\mat{3\\4\\0} + t\mat{0\\1\\3} + \mat{0\\0\\-10}$
    		    \item $\vec v = s\mat{0\\1\\0} + t\mat{0\\0\\1} + \mat{1\\-1\\1}$
    		    \item $\vec v = s\mat{1\\0} + t\mat{0\\1}$
    		    \item $\vec v = s\mat{-3\\0\\2\\1} + t\mat{-1\\1\\0\\0} + \mat{-3\\0\\3\\0}$
    		\end{enumerate}
		\end{solution}
		\prob Let $\ell_1$, $\ell_2$, and $\ell_3$ be described in vector form by
		\[
			\overbrace{\vec x=t\mat{1\\1}+\mat{1\\3}}^{\displaystyle \ell_1}
			\quad
			\overbrace{\vec x=t\mat{1\\3}+\mat{1\\1}}^{\displaystyle \ell_2}
			\quad
			\overbrace{\vec x=t\mat{2\\2}+\mat{2\\4}}^{\displaystyle \ell_3}.
		\]
		\begin{enumerate}
			\item
				Determine which pairs of the lines $\ell_1$, $\ell_2$, and $\ell_3$ intersect, 
				coincide, or are parallel.
			\item What is $\ell_1\cap\ell_2\cap\ell_3$?
		\end{enumerate}
		% Q3 Solutions
		\begin{solution} 
		    \begin{enumerate}
		        \item $\ell_1 = \ell_3$, $\ \ell_1 \& \ell_2$ intersect, \ $\ell_3 \& \ell_2$ intersect.
		        \item $\ell_1\cap\ell_2\cap\ell_3 = \Set*{\mat{2\\4}}$
		    \end{enumerate}
		\end{solution}
		\prob Let $\mathcal P_1\subseteq\R^3$ be the plane with equation $x+2y-z=3$. Let
		$\mathcal P_2$ and $\ell$ be described in vector form by
		\[
			\overbrace{\vec x=t\mat{1\\1\\1}+s\mat{0\\0\\2}+\mat{1\\3\\1}}^{\displaystyle \mathcal P_2}
			\qquad
			\overbrace{\vec x=t\mat{1\\3\\1}+\mat{1\\1\\0}}^{\displaystyle \ell}.
		\]
		\begin{enumerate}
			\item Find $\mathcal P_1\cap \ell$.
			\item Find $\mathcal P_1\cap \mathcal P_2$.
			\item Find $\mathcal P_2\cap \ell$.
			\item Give an example of a plane $\mathcal P_3$ so that
				$\mathcal P_3\cap\ell$ is empty.
			\item Does there exist a plane $\mathcal P_2'$ that is 
				parallel to $\mathcal P_2$, but which does not
				intersect $\ell$? Why or why not?
		\end{enumerate}
        % Q4 Solutions
        \begin{solution}
            \begin{enumerate}
                \item $\mathcal P_1\cap \ell = \Set*{\mat{1\\1\\0}}$
                \item $\mathcal P_1\cap \mathcal P_2 = \Set*{s\mat{1\\1\\3} + \mat{\frac{-1}{2}\\ \frac{3}{2}\\ \frac{-1}{2}} : s \in \R}$
                \item $\mathcal P_2\cap \ell = \Set*{\mat{2\\4\\1}}$
                \item $\vec v = s\mat{0\\0\\1} + t\mat{1\\3\\1}$ (Answers may vary)
                \item No. Since $P_2\cap \ell$ has one vector, the row reduction of $\mat{1 & 0 & 1 \\ 3 & 0 & 1 \\ 1 & 2 & 1 }$ has three pivots. If $P_2'$ is parallel to $P_2$, then $P_2'\cap \ell$ consists of vectors whose coordinates satisfy a system of linear equations corresponding to an augmented matrix of the form 
                \[
                \left[
                    \begin{array}{ccc|c}
                    1 & 0 & 1  & ? \\
                    3 & 0 & 1  & ? \\
                    1 & 2 & 1  & ? \\
                    \end{array}
                \right]
                \]
                , which is consistent since the now reduced form has three pivots.
            \end{enumerate}
        \end{solution}
	\prob
		Let $\vec a=\mat{1 \\1}$ and $\vec b=\mat{1 \\ -1}$. 
		The goal of this question is to produce a drawing of the set of convex linear combinations of $\vec a$ and $\vec b$.
		\begin{enumerate}
			\item Let $A$ be the set of all non-negative linear combinations of $\vec a$ and $\vec b$. Draw $A$.
			\item Let $\ell$ be the set 
			\[
				\Set{\alpha \vec a + \beta \vec b\given \alpha, \beta \in \R\text{ and } \alpha+\beta = 1}
			\]
			Rewrite $\ell$ in set-builder notation using only a single variable $t$. (Hint: Let $t$ be $\alpha$.)
			\item Justify why $\ell$ is a line, and write $\ell$ in vector form.
			\item Draw both $A$ and $\ell$ on the same grid. On a separate grid, draw $A\cap \ell$.
			\item	Write the $A \cap \ell$ in set builder notation. 
				How does $A\cap \ell$ relate to convex linear combinations?
			\item Determine the endpoints of $A \cap \ell$.
		\end{enumerate}
		\begin{solution}
			\begin{enumerate}
				\item The lines $y=x$ and $y= -x$ divide $\R^{2}$ into four
				regions. $A$ is the right-most region. 
				\item $\ell$
				is given by the set
				\[
					\Set{\vec x\in\R^2\given \vec x=t \vec a + (1-t) \vec b\text{ for some }t\in\R}.
				\]
				 \item The above set can be rewritten as
				\[
					\Set{\vec x\in\R^2\given \vec x=t (\vec a + \vec b)+\vec b\text{ for some }t\in\R}.
				\]
				 This is exactly the line given in vector form by
				\[
					\vec x = t(\vec a - \vec b)+\vec b.
				\]
				Since $\vec a-\vec b=\mat{0\\2}$, 
				$\ell$ is the vertical line containing $\vec b$ (and $\vec a$).

				\item $A \cap \ell$ is the set
				\[
					\Set{\alpha \vec a+\beta \vec b\given \alpha,\beta \geq 0\text{ and }
					\alpha+\beta = 1}.
				\]
				This is the set of convex linear
				combinations of $\vec a$ and $\vec b$. 
				\item $A
				\cap \ell$ is the line segment with endpoints $\vec a$ and $\vec b$.
			\end{enumerate}
		\end{solution}

	\prob
		Let $\vec a=\mat{2 \\0}$, $\vec b=\mat{0 \\ 2}$, and $\vec c=\mat{-1 \\ -1}$. 
		The goal of this question is to produce a drawing of the set of 
		convex linear combinations of $\vec a$, $\vec b$, and $\vec c$. This requires an understanding of the previous question.
		\begin{enumerate}
		\item \label{PROBaconvex} Let $\vec d=\mat{1 \\ 1}$. Write $\vec d$ as a convex
			linear combination of $\vec a$ and $\vec b$.
		\item \label{PROBbconvex} Let $\vec e=\mat{0 \\ 0}$. Write $\vec e$ as a convex 
			linear combination of $\vec c$ and $\vec d$.
		\item Substituting the answer to (\ref{PROBaconvex}) into the answer to part
			(\ref{PROBbconvex}), write $\vec e$ as a convex linear combination of $\vec a$, $\vec b$, and $\vec c$.
		\item Draw and label $\vec a$, $\vec b$, $\vec c$, $\vec d$, and
			$\vec e$ on the same grid.
		\item Draw the set of convex linear combinations of $\vec a$, $\vec b$, 
			and $\vec c$. Justify your answer.
		\end{enumerate}
	\begin{solution}
		\begin{enumerate}
			\item
				$
					\vec d = \tfrac{1}{2}\vec a + \tfrac{1}{2}\vec b.
				$
				 \item
				$
					\vec e= \tfrac{1}{2}\vec c+\tfrac{1}{2}\vec d.
				$
				 \item
				\begin{align*}
					\vec e & = \tfrac{1}{2}\vec c+\tfrac{1}{2}(\tfrac{1}{2}\vec a + \tfrac{1}{2}\vec b)\\
					       & = \tfrac{1}{2}\vec c+\tfrac{1}{4}\vec a + \tfrac{1}{4}\vec b.
				\end{align*}
				\item 
				\item The set is the filled-in
				triangle with vertices given by $\vec a$,
				$\vec b$, and $\vec c$. To see this, notice
				that the set of convex linear combinations of
				$\vec a$, $\vec b$, and $\vec c$ is the set of
				convex linear combinations of $\vec c$ and any
				convex linear combination of $\vec a$ and
				$\vec b$. Indeed,
				\[
				\begin{split}
					&\alpha_{1} \vec a+\alpha_{2} \vec b+ \alpha_{3}
					\vec c \\
					&\qquad= (1-\alpha_{3})\left(\tfrac{\alpha_1}{1-\alpha_3}\vec
					a + \tfrac{\alpha_2}{1-\alpha_3}\vec b\right)+\alpha_{3}
					\vec c,
				\end{split}
				\]
				 and $\alpha_{1} \vec a+\alpha_{2} \vec b+ \alpha_{3}
				\vec c$ is a convex linear combination of $\vec
				a$, $\vec b$, and $\vec c$ exactly when
				\[
					\frac{\alpha_1}{1-\alpha_3}\vec a + \frac{\alpha_2}{1-\alpha_3}\vec
					b
				\]
				 is a convex linear combination of $\vec a$ and $\vec
				b$ (verify this!).

				By the previous part, the set of convex linear combinations
				of $\vec a$ and $\vec b$ is the line segment
				between $\vec a$ and $\vec b$. Call this line segment $S$.
				Now we know the set of convex linear
				combinations of $\vec a$, $\vec b$, and $\vec c$
				is the union of every line segment from $\vec c$
				and a vector in $S$. This is the solid triangle with
				vertices given by $\vec a$, $\vec b$, and $\vec	c$.
		\end{enumerate}
	\end{solution}

	\prob
		Let $\vec x=\mat{1 \\ 1}$, $\vec y = \mat{3 \\ -1}$ and $\vec z=\mat{-2 \\ -2}.$  Draw the following subsets of $\R^2$.
		\begin{enumerate}
		\item All non-negative linear combinations of $\vec x$ and $\vec y$.
		\item All non-negative linear combinations of $\vec x$ and $\vec z$.
		\item\label{PROBcconvex} All convex linear combinations of $\vec y$ and $\vec z$.
		\item\label{PROBdconvex} All convex linear combinations of $\vec x$ and $\vec z$.
		\item All convex linear combinations of $\vec x$, $\vec y$ and $\vec z$.

		\end{enumerate}
	\begin{solution}
		\begin{enumerate}
			\item The acute angled section of $\R^{2}$ between the ray
				from the origin through $\mat{1 \\ 1}$ and the ray
				from the origin through $\mat{3 \\ -1}$. 
			\item
				The line through the origin and
				$\mat{1 \\ 1}$. 
			\item The line segment between $\mat{3
				\\ -1}$ and $\mat{-2 \\ -2}.$ 
			\item The line segment
				between $\mat{1 \\ 1}$ and $\mat{-2 \\ -2}.$ 
			\item
				The filled-in triangle with vertices $\mat{1\\1}$,
				$\mat{3 \\ -1}$ and $\mat{-2 \\ -2}.$
		\end{enumerate}
	\end{solution}

	\prob Describe the sets in (\ref{PROBcconvex}) and (\ref{PROBdconvex}) in set builder notation.
	\begin{solution}
		(\ref{PROBcconvex}) 
		\[
			\Set*{\vec v \in \R^{2}\given \vec v=t\mat{-5\\-1}+\mat{3\\-1}\text{ for some }0\leq t\leq 1}.
		\]
		
		(\ref{PROBdconvex}) 
		\[
			\Set*{\vec w \in \R^{2}\given \vec w=t\mat{-3\\-3}+\mat{1\\1}\text{ for some }0\leq t\leq 1}.
		\]
	\end{solution}

	\prob
		Determine if the points $P=(-2,0)$ and $Q=(0,-2)$ are convex linear 
		combinations of the vectors $\vec u =\mat{1 \\4}$, $\vec v =\mat{-5 \\8}$, 
		and $\vec w =\mat{-2 \\-6}$. First solve this question by drawing a picture. Then justify algebraically.
	\begin{solution}
		Geometrically, the set of convex linear combinations of $\vec u$, $\vec v$, and $\vec w$ is a
		filled-in triangle with vertices at $\vec u, \vec v$ and $\vec w$. The point
		$P$ lies inside this triangle, while $Q$ does not.
		
		To argue algebraically, suppose 
		$P=t_{1}\vec u + t_{2}\vec v + t_{3} \vec w$ and
		$t_{1}+t_{2}+t_{3}=1$. From these assumptions, we can set up a system of equations
		which has a unique solution
		\[
			P=\tfrac14 \vec u + \tfrac14 \vec v + \tfrac12 \vec w,
		\]
		and so $P$ is a convex linear combination of $\vec u$, $\vec v$, and $\vec w$.
		 The same procedure with $Q$ gives a unique solution with coefficients
		$t_{1}=\frac 59, t_{2}=-\frac 19, t_{3}=\frac 59$, and so $Q$ is
		not a convex linear combination of $\vec u$, $\vec v$, and $\vec w$.

	\end{solution}
	\end{problist}
\end{exercises}
