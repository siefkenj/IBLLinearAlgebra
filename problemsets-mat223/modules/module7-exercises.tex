\begin{exercises}
	\begin{problist}

		\prob Each system of equations below concerns the variables $x$, $y$, and $z$.
		Rewrite each system as a single matrix equation.
		\begin{enumerate}
			\item $\systeme{
											x-y+z=1,
											2x-y+z=2,
											3x+y-z=3
										}$
			\item $\systeme{
											x+z=6
										}$
			\item $\systeme{
											5x-9y+2z=0,
											-y=1
										}$
		\end{enumerate}
		\begin{solution}
			\begin{enumerate}
				\item $\mat{1&-1&1\\2&-1&1\\3&1&-1}\mat{x\\y\\z} = \mat{1\\2\\3}$
				\item $\mat{1&0&1}\mat{x\\y\\z} = \mat{6}$
				\item $\mat{5&-9&2\\0&-1&0}\mat{x\\y\\z} = \mat{0\\1}$
			\end{enumerate}
		\end{solution}

		\prob Find all vectors orthogonal to:
		\begin{enumerate}
			\item $\mat{1\\2\\3}$ and $\mat{2\\2\\3}$
			\item $\mat{0\\5\\6}$ and $\mat{1\\10\\2}$
			\item $\mat{1\\0\\0}$, $\mat{0\\1\\0}$, and $\mat{0\\0\\1}$
			\item $\mat{2\\6\\-1}$
		\end{enumerate}
		\begin{solution}
			\begin{enumerate}
				\item \label{PARTA} If $\vec{x} = \mat{x\\y\\z}$ is orthogonal to $\mat{1\\2\\3}$ and $\mat{2\\2\\3}$, then
				\begin{align*}
					\vec x \cdot \mat{1\\2\\3} &= 0 \quad\implies\quad x+2y+3z = 0 \\
					\vec x \cdot \mat{2\\2\\3} &= 0 \quad\implies\quad 2x+2y+3z = 0.
				\end{align*}
				This means we need to solve the system
				\[
					\underbrace{\mat{1&2&3\\2&2&3}}_{A} \mat{x\\y\\z} = \mat{0\\0}.
				\]
				Row reducing $A$ yields
				\[
					\rref(A) = \mat{1&0&0\\0&1&2/3},
				\]
				so third column corresponds to a free variable. Let $z=t$, then $x=0$ and $y = -\frac{3t}{2}$. 
				The complete solution expressed in vector form is
				\[
					\vec x = t\mat{0\\-3/2\\1} \quad \text{or} \quad \vec x = t\mat{0\\-3\\2}.
				\]
				\item  Proceeding as in \ref{PARTA}, we need to solve the matrix equation
				\[
					\underbrace{\mat{0&5&6\\1&10&2} }_{B} \underbrace{\mat{x\\y\\z}}_{\vec{x}} = \mat{0\\0}.
				\]
				We obtain
				\[
					\rref (B) = \mat{1&0&-10\\0&1&6/5},
				\]
				so the complete solution is 
				\[
					\vec x = t\mat{10\\-6/5\\1} \quad \text{or} \quad \vec x = t\mat{50\\-6\\5}.
				\]


				\item We need to solve the matrix equation
				\[
					\mat{1&0&0\\0&1&0\\0&0&1} \mat{x\\y\\z} = \mat{0\\0\\0}.
				\]
				The only solution is $\vec x = \vec 0$.


				\item We need to solve the matrix equation
				\[
					\underbrace{\mat{2&6&-1} }_{A} \underbrace{\mat{x\\y\\z}}_{\vec x} = \mat{0}.
				\]
				Row reducing, we obtain
				\[
					\rref (A) = \mat{1&3&-1/2}.
				\]
				Both the second and third columns correspond to free variables. Let $s=y$ and $t=z$, then $x = -3s + \frac{t}{2}$, 
					so we have
				\[
					\mat{x\\y\\z} = \matc{-3s+t/2\\s\\t} = s\mat{-3\\1\\0} + t\mat{1/2\\0\\1}.
				\]
				Therefore, the complete solution is
				\[
					\vec x = s\mat{-3\\1\\0} + t\mat{1/2\\0\\1}.
				\]
			\end{enumerate}
		\end{solution}	

		\prob Express each plane or hyperplane in normal form.
			\begin{enumerate}
				\item $\vec x = t\mat{0\\2\\2}+s\mat{1\\1\\1}+\mat{1\\0\\3}$
				\item $\vec x = t\mat{1\\6\\8}+s\mat{2\\0\\2}+\mat{0\\0\\9}$
				\item $\vec x = t\mat{1\\5\\15\\20}+s\mat{3\\0\\35\\59}+
				r\mat{1\\4\\0\\18}+\mat{1\\6\\0\\0}$
			\end{enumerate}
		\begin{solution}
			\emph{Key fact to remember}: a normal vector is orthogonal to the direction vectors.
			\begin{enumerate}
				\item  
					If $\vec{n} = \mat{x\\y\\z}$ is a normal vector, then
					\begin{align*}
						\vec x \cdot \mat{0\\2\\2} &= 0  \quad\implies\quad 2y+2z = 0\\
						\vec x \cdot \mat{1\\1\\1} &= 0 \quad\implies\quad x+y+z = 0.
					\end{align*}
					Expressed in matrix form, this system becomes 
					\[
						\underbrace{\mat{0&2&2\\1&1&1}}_{A} \mat{x\\y\\z} = \mat{0\\0}.
					\]
					Row reduction yields
					\[
						\Rref(A) = \mat{1&0&0\\0&1&1},
					\]
					so complete solution in vector form is 
					\[
						\vec x = t\mat{0\\-1\\1}.
					\]
					This means that any non-zero multiple of $\mat{0&-1&1}$ is a normal vector for this plane, so the normal form of the plane is 
					\[
						\mat{0&-1&1} \cdot \left(\mat{x\\y\\z} - \mat{1\\0\\3}\right)=0.
					\]
				\item  We first wish to find all vectors $\vec x$ orthogonal to $\mat{1\\6\\8}$ and $\mat{2\\0\\2}$. 
					To do this, we solve the system 
					$\mat{1&6&8\\2&0&2}\mat{x\\y\\z}=\mat{0\\0}.$

					Row reducing, we have 
					\begin{align*}
						\mat{1&6&8\\2&0&2}&\rightarrow \mat{1&6&8\\0&-12&-14}\\
						&\rightarrow \mat{1&6&8\\0&6&7}\\
						&\rightarrow \mat{1&0&1\\0&6&7}\\
					\end{align*}
					which has nullspace equal to $\Span{\mat{6\\7\\-6}}$.
					Since $\mat{0\\0\\9}$ is a point on the plane, we then get that the normal form of the plane is \[\mat{6\\7\\-6}\cdot \left(\mat{x\\y\\z}-\mat{0\\0\\9}\right)=0.\]
					\item We first wish to find all vectors $\vec x$ orthogonal to $\mat{1\\5\\15\\20}$, $\mat{3\\0\\35\\59}$, and $\mat{1\\4\\0\\18}$. 
						To do this we solve the system \[\mat{1&5&15&20\\3&0&35&59\\1&4&0&18}\mat{x\\y\\z\\w}=\mat{0\\0\\0}.\] 
						Row reducing, we have 
						\begin{align*}
							\mat{1&5&15&20\\3&0&35&59\\1&4&0&18}&\rightarrow \mat{1&5&15&20\\3&0&35&59\\0&-1&-15&-2}\\
							&\rightarrow \mat{1&5&15&20\\0&-15&-10&-1\\0&-1&-15&-2}\\
							&\rightarrow \mat{1&5&15&20\\0&15&10&1\\0&1&15&2}\\
							&\rightarrow \mat{1&5&15&20\\0&1&15&2\\0&15&10&1}\\
							&\rightarrow \mat{1&5&15&20\\0&1&15&2\\0&0&-215&-29}\\
							&\rightarrow \mat{1&0&-60&10\\0&1&15&2\\0&0&-215&-29}\\
							&\rightarrow \mat{1&0&-60&10\\0&1&15&2\\0&0&215&29}\\
							&\rightarrow \mat{1&0&-60&10\\0&1&0&-\frac{1}{43}\\0&0&215&29}\\
							&\rightarrow \mat{1&0&0&\frac{778}{43}\\0&1&0&-\frac{1}{43}\\0&0&215&29}\\
							&\rightarrow \mat{1&0&0&\frac{778}{43}\\0&1&0&-\frac{1}{43}\\0&0&1&\frac{29}{215}}
						\end{align*}
						which has nullspace equal to $\Span{\mat{3890\\-5\\29\\-215}}$. Thus, since $\mat{1\\6\\0\\0}$ is a point on the hyperplane, 
						we then get that the normal form of the hyperplane is \[\mat{3890\\-5\\29\\-215}\cdot\left(\mat{x\\y\\z\\w}-\mat{1\\6\\0\\0} \right)=0.\]
				\end{enumerate}
			\end{solution}	

		\prob
		Let $\mathcal P=
			\Set{(x,y,z)\given 2x+4y-4z=7}
		$,
		let $\mathcal Q$ be the plane specified in vector form by
		\[
			\vec x = t\mat{-1\\2\\0}+s\mat{5\\0\\2}+\mat{0\\7\\1},
		\]
		and let $\mathcal R$ be the plane specified in normal form by
		\[
			\mat{2\\-8\\2} \cdot \left(\mat{x\\y\\z}-\mat{1\\7\\0}\right).
		\]
		Find $\mathcal P\cap\mathcal Q\cap\mathcal R$.
		\begin{solution}
			We first express all the planes in the same form (any form works but we choose to use the equation form).
			$\mathcal{P}$ is already expressed in this form: $2x+4y-4z=7$.\\
			We are given $\mathcal{Q}$ in the form \[\vec x=t\mat{-1\\2\\0}+s\mat{5\\0\\2}+\mat{0\\7\\1}\].
			Writing this into vector form yields
			\begin{align*}
				x&=-t+5s\\
				y&=2t+7\\
				z&=2s+1
			\end{align*}
			then substituting the above values yields 
			\begin{align*}
				2x&=-2t+5(2s)\\
				&=-(y-7)+5(z-1)\\
				2x+y-5z&=2.
			\end{align*}
			Thus, having eliminated $t$ and $s$ from our original equations we find that the equation form of $\mathcal{Q}$ is $2x+y-5z=2$.
			We are given $\mathcal{R}$ in the form \[\mat{2\\-8\\2}\cdot \left(\mat{x\\y\\z}-\mat{1\\7\\0}\right)=0\]
			which, expanding the dot product, yields 
			\begin{align*}
				\mat{2\\-8\\2}\cdot \mat{x\\y\\z}&=\mat{2\\-8\\2}\cdot\mat{1\\7\\0}\\
				2x-8y+2z&=2-7(8)\\
				2x-8y+2z&=-54.
			\end{align*}
			Thus the equation form of $\mathcal{R}$ is $2x-8y+2z=-54$.
			Having written all three planes in equation form, finding $\mathcal{P}\cap\mathcal{Q}\cap\mathcal{R}$ is the same as 
			solving the matrix equation \[\mat{2&4&-4\\2&1&-5\\2&-8&2}\mat{x\\y\\z}=\mat{7\\2\\-54}.\] Row-reducing on both sides yields 
			\begin{align*}
				\mat{2&4&-4\\2&1&-5\\2&-8&2}\mat{x\\y\\z}=\mat{7\\2\\-54}&\rightarrow \mat{2&4&-4\\0&-3&-1\\2&-8&2}\mat{x\\y\\z}=\mat{7\\-5\\-54}\\
				&\rightarrow \mat{2&4&-4\\0&-3&-1\\0&-12&6}\mat{x\\y\\z}=\mat{7\\-5\\-61}\\
				&\rightarrow \mat{2&0&-\frac{16}{3}\\0&-3&-1\\0&-12&6}\mat{x\\y\\z}=\mat{\frac{1}{3}\\-5\\-61}\\
				&\rightarrow \mat{2&0&-\frac{16}{3}\\0&-3&-1\\0&0&10}\mat{x\\y\\z}=\mat{\frac{1}{3}\\-5\\-41}\\
				&\rightarrow \mat{2&0&-\frac{16}{3}\\0&-3&-1\\0&0&1}\mat{x\\y\\z}=\mat{\frac{1}{3}\\-5\\-\frac{41}{10}}\\
				&\rightarrow \mat{2&0&0\\0&-3&-1\\0&0&1}\mat{x\\y\\z}=\mat{-\frac{323}{15}\\-5\\-\frac{41}{10}}\\
				&\rightarrow \mat{2&0&0\\0&-3&0\\0&0&1}\mat{x\\y\\z}=\mat{-\frac{323}{15}\\-\frac{91}{10}\\-\frac{41}{10}}\\
				&\rightarrow \mat{1&0&0\\0&1&0\\0&0&1}\mat{x\\y\\z}=\mat{-\frac{323}{30}\\ \frac{91}{30}\\-\frac{41}{10}}
			\end{align*}
			so that $\mathcal{P}\cap\mathcal{Q}\cap\mathcal{R}$ consists of exactly one point. 
			\[ \mathcal{P}\cap\mathcal{Q}\cap\mathcal{R}=\left\lbrace\mat{-\frac{323}{30}\\ \frac{91}{30}\\-\frac{41}{10}}\right\rbrace.\]
		\end{solution}	
	\end{problist}
\end{exercises}
