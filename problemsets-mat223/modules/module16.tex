
Suppose $\mathcal T$ is a linear transformation and $\vec v_1$ and $\vec v_2$ are eigenvectors
with eigenvalues $\lambda_1$ and $\lambda_2$. With this setup, for any $\vec a\in\Span\Set{\vec v_1,\vec v_2}$,
we can compute $\mathcal T(\vec a)$ with minimal effort.

Let's get specific. Define $\mathcal T:\R^2\to\R^2$
to be the linear transformation with matrix $M=\mat{1&2\\3&2}$. Let $\vec v_1=\mat{-1\\1}$ and $\vec v_2=\mat{2\\3}$,
and notice that $\vec v_1$ is an eigenvector for $\mathcal T$ with eigenvalue $-1$ and that $\vec v_2$ is an
eigenvector for $\mathcal T$ with eigenvalue $4$. Let $\vec a=\vec v_1+\vec v_2$.

Now,
\[
	\mathcal T(\vec a)=\mathcal T(\vec v_1+\vec v_2)=\mathcal T(\vec v_1)+\mathcal T(\vec v_2)=-\vec v_1+4\vec v_2.
\]

We didn't need to refer to the entries of $M$ to compute $\mathcal T(\vec a)$.

Let's explore further. Let $\mathcal V=\Set{\vec v_1,\vec v_2}$ and notice that $\mathcal V$ is a basis
for $\R^2$. By definition $[\vec a]_{\mathcal V}=\mat{1\\1}$, and so we just computed
\[
	\mathcal T\mat{1\\1}_{\mathcal V} = \mat{-1\\4}_{\mathcal V}.
\]
When represented in the $\mathcal V$ basis, computing $\mathcal T$ is easy. In general,
\[
	\mathcal T(\alpha\vec v_1+\beta\vec v_2)=\alpha\mathcal T(\vec v_1)+\beta\mathcal T(\vec v_2)=-\alpha\vec v_1+4\beta\vec v_2,
\]
and so
\[
	\mathcal T\mat{\alpha\\\beta}_{\mathcal V} = \mat{-\alpha\\4\beta}_{\mathcal V}.
\]
In other words, $\mathcal T$, when acting on vectors written in the $\mathcal V$ basis, just multiplies each coordinate
by an eigenvalue. This is enough information to determine the matrix for $\mathcal T$ in the $\mathcal V$ basis:
\[
	[\mathcal T]_{\mathcal V}=\mat{-1&0\\0&4}.
\]

The matrix representations $[\mathcal T]_{\mathcal E}=\mat{1&2\\3&2}$ and $[\mathcal T]_{\mathcal V}=\mat{-1&0\\0&4}$
are equally valid, but writing $\mathcal T$ in the $\mathcal V$ basis gives a very simple matrix!

\Heading{Diagonalization}

Recall that two matrices are similar if they represent the same transformation but in possibly different bases.
The process of \emph{diagonalizing} a matrix $A$ is that of finding a diagonal matrix that is similar to $A$,
and you can bet that this processes is closely related to eigenvectors/values.

Let $\mathcal T:\R^n\to\R^n$ be a linear transformation and suppose that $\mathcal B=\Set{\vec b_1,\ldots,\vec b_n}$
is a basis so that
\[
	[\mathcal T]_{\mathcal B} = \matc{\alpha_1&0&\cdots &0\\0&\alpha_2&\cdots &0\\\vdots&\vdots&\ddots&\vdots\\0&0&\cdots&\alpha_n}
\]
is a diagonal matrix. This means that $\vec b_1,\ldots,\vec b_n$ are eigenvectors for $\mathcal T$! The proof goes as follows:
\[
	[\mathcal T]_{\mathcal B}[\vec b_1]_{\mathcal B} = 
	\matc{\alpha_1&0&\cdots &0\\0&\alpha_2&\cdots &0\\\vdots&\vdots&\ddots&\vdots\\0&0&\cdots&\alpha_n}
	\matc{1\\0\\\vdots\\0} = \matc{\alpha_1\\0\\\vdots\\0}=\alpha_1[\vec b_1]_{\mathcal B}=[\alpha_1\vec b_1]_{\mathcal B},
\]
and in general
\[
	[\mathcal T]_{\mathcal B}[\vec b_i]_{\mathcal B} = \alpha_i[\vec b_i]_{\mathcal B}=[\alpha_i\vec b_i]_{\mathcal B}.
\]
Therefore, for $i=1,\ldots,n$, we have
\[
	\mathcal T\vec b_i=\alpha_i\vec b_i.
\]
Since $\mathcal B$ is a basis, $\vec b_i\neq \vec 0$ for any $i$, and so each $\vec b_i$ is an eigenvector for $\mathcal T$ with
corresponding eigenvalue $\alpha_i$.

We've just shown that if a linear transformation $\mathcal T:\R^n\to\R^n$ can be represented by a diagonal matrix,
then there must be a basis for $\R^n$ consisting of eigenvectors for $\mathcal T$. The converse is also true.

Suppose again that $\mathcal T:\R^n\to\R^n$ is a linear transformation and that $\mathcal B=\Set{\vec b_1,\ldots,\vec b_n}$
is a basis of eigenvectors for $\mathcal T$ with corresponding eigenvalues $\alpha_1,\ldots,\alpha_n$. By definition,
\[
	\mathcal T(k\vec b_i)=k\mathcal T(\vec b_i)=\alpha_ik\vec b_i,
\]
and so
\[
	\mathcal T\matc{k_1\\k_2\\\vdots\\k_n}_{\mathcal B} = \matc{\alpha_1k_1\\\alpha_2k_2\\\vdots\\\alpha_nk_n}_{\mathcal B}
	\qquad\text{which is equivalent to}\qquad
	[\mathcal T]_{\mathcal B}\matc{k_1\\k_2\\\vdots\\k_n} = \matc{\alpha_1k_1\\\alpha_2k_2\\\vdots\\\alpha_nk_n}.
\]
The only matrix that does this is
\[
	[\mathcal T]_{\mathcal B} = \matc{\alpha_1&0&\cdots &0\\0&\alpha_2&\cdots &0\\\vdots&\vdots&\ddots&\vdots\\0&0&\cdots&\alpha_n},
\]
which is a diagonal matrix.

What we've shown is summarized by the following theorem.
\begin{theorem}
	A linear transformation $\mathcal T:\R^n\to\R^n$ can be represented by a diagonal matrix if and
	only if there exists a basis for $\R^n$ consisting of eigenvectors for $\mathcal T$. If
	$\mathcal B$ is such a basis, then $[\mathcal T]_{\mathcal B}$ is a diagonal matrix.
\end{theorem}

Now that we have a handle on representing a linear transformation by a diagonal matrix, let's tackle
the problem of diagonalizing a matrix itself.

\SavedDefinitionRender{Diagonalizable}

Suppose $A$ is an $n\times n$ matrix. $A$ induces some transformation $\mathcal T_A:\R^n\to\R^n$. By definition,
this means $A=[\mathcal T_A]_{\mathcal E}$. The matrix $B$ is similar to $A$ if there is some basis $\mathcal V$ so that
$B=[\mathcal T_A]_{\mathcal V}$. Using change-of-basis matrices, we see
\[
	A=\BasisChange{\mathcal V}{\mathcal E}[\mathcal T_A]_{\mathcal V}\BasisChange{\mathcal E}{\mathcal V}
	=\BasisChange{\mathcal V}{\mathcal E}B\BasisChange{\mathcal E}{\mathcal V}.
\]
In other words, $A$ and $B$ are similar if there is some invertible change-of-basis matrix $P$ so
\[
	A=PBP^{-1}.
\]
Based on our earlier discussion, $B$ will be a diagonal matrix if and only if $P$ is the change-of-basis matrix for a
basis of eigenvectors. In this case, we know $B$ will be the diagonal matrix with eigenvalues along the diagonal (in the proper order).

\begin{example}
	Let $A=\mat{1&2&5\\-11&14&5\\-3&2&9}$ be a matrix and notice that $\vec v_1=\mat{5\\5\\1}$, $\vec v_2=\mat{1\\1\\1}$,
	and $\vec v_3=\mat{1\\3\\1}$ are eigenvectors for $A$. Diagonalize $A$.

	First, we find the eigenvalues that correspond to the eigenvectors $\vec v_1, \vec v_2$, 
	and $\vec v_3$. 
	
	$A\vec v_1=\mat{5\\5\\1}=4\vec v_1$, $A\vec v_2=\mat{8\\8\\8}=8\vec v_2$, and
	$A\vec v_3=\mat{12\\36\\12}=12\vec v_3$. 
	
	The eigenvalue of $\vec v_1$ is 4, the eigenvalue of $\vec v_2$ is 8, 
	and the eigenvalue of $\vec v_3$ is 12.
	
	The change-of-basis matrix which converts from the basis of eigenvalues to the standard basis is
	\[
		P = \mat{5&1&1\\5&1&3\\1&1&1}.
	\]
	
	The change-of-basis matrix which converts from the standard basis to the basis
	of eigenvalues is
	\[
		P^{-1} = \mat{\frac{1}{4}&0&-\frac{1}{4}\\\frac{1}{4}&-\frac{1}{2}&\frac{5}{4}\\-\frac{1}{2}&\frac{1}{2}&0}.
	\]
	The matrix $A$ written in the basis of eigenvectors is
	\[
		B = \mat{4&0&0\\0&8&0\\0&0&12}.
	\]
	Thus
	\[
		A = PBP^{-1}= \mat{5&1&1\\5&1&3\\1&1&1}\mat{4&0&0\\0&8&0\\0&0&12}
		\mat{\frac{1}{4}&0&-\frac{1}{4}\\\frac{1}{4}&-\frac{1}{2}&\frac{5}{4}\\-\frac{1}{2}&\frac{1}{2}&0}.
	\]
\end{example}

\Heading{Non-diagonalizable Matrices}

Is every matrix diagonalizable? Unfortunately the world is not that sweet. But, we have a tool to tell if a matrix is
diagonalizable---checking to see if there is a basis of eigenvectors.

\begin{example}
	Is the matrix $R=\mat{0&-1\\1&0}$ diagonalizable?

	Computing, $\Char(R)=\lambda^2+1$ which has no real roots. Therefore,
	$R$ has no real eigenvalues. Consequently, $R$ has no real eigenvectors,
	and so $R$ is not diagonalizable\footnote{ If we allow complex eigenvalues, then $R$ \emph{is}
	diagonalizable and is similar to the matrix $\mat{i&0\\0&-i}$. So, to be more precise, we
	might say $R$ is not \emph{real} diagonalizable.}.
\end{example}

\begin{example}
	Is the matrix $D=\mat{5&0\\0&5}$ diagonalizable?

	For every vector $\vec v\in \R^2$, we have $D\vec v=5\vec v$, and so every
	non-zero vector in $\R^2$ is an eigenvector for $D$. Thus, $\mathcal E=\Set{\xhat, \yhat}$
	is a basis of eigenvectors for $\R^2$, and so $D$ is diagonalizable\footnote{ Of course, every
	square matrix is similar to itself and $D$ is already diagonal, so of course it's diagonalizable.}.
\end{example}

\begin{example}
	Is the matrix $J=\mat{5&1\\0&5}$ diagonalizable?

	The eigenvectors of $J$ all lie on $\Span\Set*{\mat{1\\0}}$. Therefore it does not admit a basis of
	eigenvectors and thus, is not diagonalizable.
\end{example}

\begin{example}
	Is the matrix $K=\mat{5&1\\0&2}$ diagonalizable?
	
	$K$ admits a basis of eigenvectors. for example $\Set*{\mat{1\\0},\mat{-1\\3}}$. Thus, $K$ is
	diagonalizable.
\end{example}

\begin{emphbox}[Takeaway]
	Not all matrices are diagonalizable, but you can check if an $n\times n$
	matrix is diagonalizable by determining whether there is a basis of eigenvectors for $\R^n$.
\end{emphbox}

\Heading{Geometric and Algebraic Multiplicities}

When analyzing linear transformations or matrices, we're often interested in
studying the subspaces where vectors are stretched by only one eigenvalue. These 
are called the \emph{eigenspaces}.

\SavedDefinitionRender{Eigenspace}

Now is the time when linear algebra and regular algebra (the solving of non-linear equations)
combine. We know, every root of the characteristic polynomial of a matrix gives an eigenvalue
for that matrix. Since the degree of the characteristic polynomial of an $n\times n$ matrix
is always $n$, the fundamental theorem of algebra tells us exactly how many roots to expect.

Recall that the \emph{multiplicity} of a root of a polynomial is the power of that root
in the factored polynomial. So, for example $p(x)=(4-x)^3(5-x)$ has a root of $4$ with multiplicity
$3$ and a root of $5$ with multiplicity $1$.

\begin{example}
	Let $R=\mat{0&-1\\1&0}$ and find the geometric and algebraic multiplicity of each eigenvalue of $R$.
	
	Computing, $\Char(R)=\lambda^2+1$ which has no real roots. Therefore,
	$R$ has no real eigenvalues.\footnote{ If we allow complex eigenvalues, then the eigenvalues
	$i$ and $-i$ both have geometric and algebraic multiplicity of 1.}
\end{example}

\begin{example}
	Let $D=\mat{5&0\\0&5}$ and find the geometric and algebraic multiplicity of each eigenvalue of $D$.
	
	$\Char(D)=(5-\lambda)^2$, so 5 is an eigenvalue of $D$ with algebraic multiplicity 2.
	The eigenspace of $D$ corresponding to 5 is $\R^2$. Thus, the geometric multiplicity of 5 is 2.
\end{example}

\begin{example}
	Let $J=\mat{5&1\\0&5}$ and find the geometric and algebraic multiplicity of each eigenvalue of $J$.
	
	$\Char(J)=(5-\lambda)^2$, so 5 is an eigenvalue of $J$ with algebraic multiplicity 2.
	The eigenspace of $J$ corresponding to 5 is $\Span\Set*{\mat{1\\0}}$. Thus, the geometric 
	multiplicity of 5 is 1.
\end{example}

\begin{example}
	Let $K=\mat{5&1\\0&2}$ and find the geometric and algebraic multiplicity of each eigenvalue of $K$.
	
	$\Char(K)=(5-\lambda)(2-\lambda)$, so 5 and 2 are eigenvalues of $K$, both with algebraic multiplicity 1.
	The eigenspace of $K$ corresponding to 5 is $\Span\Set*{\mat{1\\0}}$ and the eigenspace
	corresponding to 2 is $\Span\Set*{\mat{-1\\3}}$. Thus, both 5 and 2 have a geometri 
	multiplicity of 1.
\end{example}


Consider the following two theorems.

\begin{theorem}(Fundamental Theorem of Algebra)
	Let $p$ be a polynomial of degree $n$. Then, if complex roots are allowed,
	the sum of the multiplicities of the roots of $p$ is $n$.
\end{theorem}

\begin{theorem}
	Let $\lambda$ be an eigenvalue of the matrix $A$. Then
	\[
		\text{geometric mult}(\lambda)\leq \text{algebraic mult}(\lambda).
	\]
\end{theorem}

We can now deduce the following.
\begin{theorem}
	An $n\times n$ matrix $A$ is diagonalizable if and only if the sum of its geometric multiplicities
	is equal to $n$. Further, provided complex eigenvalues are permitted, $A$ is diagonalizable if and
	only if all its geometric multiplicities equal its algebraic multiplicities.
\end{theorem}
\begin{proof}
	XXX Finish
\end{proof}
