\begin{exercises}
	\begin{problist}
		\prob Let $\mathcal A=\Set{\vec a_1,\vec a_2}$ where $\vec a_1=\mat{2\\1}_{\mathcal E}$ and $\vec a_2=\mat{1\\-2}_{\mathcal E}$ and let $\mathcal B=\Set{\vec b_1,\vec b_2}$ where $\vec b_1=\mat{3\\-1}_{\mathcal E}$ and $\vec b_2=\mat{-2\\3}_{\mathcal E}$ be bases for $\R^2$. 
	    \begin{enumerate}
	        \item Given that $[\vec x]_{\mathcal A}=\mat{1\\-1}$, find $[\vec x]_{\mathcal B}$.
	        \item Find the change of basis matrix $\BasisChange{\mathcal A}{\mathcal B}$.
	    \end{enumerate}
	    
	    \prob Let $\mathcal A=\Set{\vec a_1,\vec a_2,\vec a_3}$ where $\vec a_1=\mat{2\\1\\0}_{\mathcal E}$, $\vec a_2=\mat{1\\-2\\0}_{\mathcal E}$, and $\vec a_3=\mat{0\\0\\1}_{\mathcal E}$.
	    
	    Let $\mathcal B=\Set{\vec b_1,\vec b_2,\vec b_3}$ where $\vec b_1=\mat{3\\0\\-1}_{\mathcal E}$, $\vec b_2=\mat{-2\\0\\3}_{\mathcal E}$, and $\vec b_3=\mat{0\\1\\0}_{\mathcal E}$.
	    
	    Then  both $\mathcal A$ and $\mathcal B$ are bases for $\R^3$. 
	    \begin{enumerate}
	        \item Given that $[\vec x]_{\mathcal A}=\mat{1\\1\\1}$, find $[\vec x]_{\mathcal B}$.
	        \item Find the change of basis matrix $\BasisChange{\mathcal A}{\mathcal B}$.
	    \end{enumerate}
	    
	    \prob Let $\mathcal B=\Set{\vec b_1,\vec b_2}$ where $\vec b_1=\mat{1\\1}_{\mathcal E}$ and $\vec b_2=\mat{-1\\1}_{\mathcal E}$ be a basis for $\R^2$.
	    
	    For each $\mathcal T:\R^2\to\R^2$ defined below, compute $[\mathcal T]_{\mathcal E}$ and $[\mathcal T]_{\mathcal B}$.
	        \begin{enumerate}
	            \item   $\mathcal T$ is the linear transformation that doubles every vector.
			    \item   $\mathcal T$ is the linear transformation that rotates every vector counter clockwise by $45^\circ$.
			    \item   $\mathcal T$ is the linear transformation that projects every vector onto the $y$-axis.
			    \item   $\mathcal T$ is the linear transformation that reflects every vector over the line $y=x$.
	        \end{enumerate}
        
        \prob For each statement listed below, indicate whether it is correct or incorrect statement. Justify you answer.
        \begin{enumerate}
            \item   Any invertible $n \times n$ matrix can be viewed as a change of basis matrix.
            \item   Any $n \times n$ matrix is similar to itself.
            \item   Let $A$ be an arbitrary $m \times n$ matrix. If $m \neq n$, then we are unable to find a similar matrix for $A$.
            \item   Any invertible $n \times n$ matrix $A$ is similar to $A^{-1}$ since $AA^{-1}=I$.
        \end{enumerate}
	\end{problist}
\end{exercises}
