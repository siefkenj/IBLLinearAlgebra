\begin{exercises}
		% Topics:
		% Sets, set builder notation, set operations,
		% vectors \& scalars, vector notation, vectors \& points, vector arithmetic,
		% coordinates \& the standard basis, higher dimensions,
	\begin{problist}
		% Computation (4 questions)
		\prob
		\begin{enumerate}
			\item
				Write the following vectors as column vectors.
				\begin{enumerate}
					\item $4\xhat -3\zhat +2\yhat -2\xhat\in\R^3$.
					\item $1\yhat +\xhat -5\yhat \in\R^2$.
				\end{enumerate}
			\item
				Write the following vectors as linear combinations of the standard
				basis vectors $\sbasis{1}$, $\sbasis{2}$, $\sbasis{3}$, and
				$\sbasis{4}$:
				\begin{enumerate}
					\item $\mat{0\\-3\\2}$
					\item $\displaystyle
						\mat{-2\\5\\4} + (-1)\mat{-1\\2\\ 5} + \frac12\mat{2\\0\\2}$
					\item $\displaystyle
						2\mat{-1\\-2\\0\\ 5}
						+1\mat{3\\4\\-2\\ 1}
						+\left(-\frac13\right)\mat{0\\-3\\0\\ 6}$
				\end{enumerate}
		\end{enumerate}
		% Q1 Solution
		\begin{solution}
		    \begin{enumerate}
		        \item 
		            \begin{enumerate}
		                \item $\mat{2\\2\\-3}$
		                \item $\mat{1\\-4}$
		            \end{enumerate}
		        \item
		            \begin{enumerate}
		                \item $\xhat - 2\yhat + 3\zhat$
		                \item $3\yhat$
		            \end{enumerate}
		    \end{enumerate}
		\end{solution}

		\prob
		Compute
		\[
			3\mat{2\\-1\\1\\1\\0}+
			(-2)\mat{1\\2\\-7\\3\\0}+
			\mat{-3\\3\\9\\2\\2}
		\]
		% Q2 Solutions
		\begin{solution}
		    $\mat{1\\-4\\26\\-1\\2}$
		\end{solution}

		\prob[\hefferon[2.21,2.22]]
			Decide if the vector is in the set. If it is, what value of the
			parameters produce that vector?
			\begin{enumerate}
				\item
					$\mat{5\\-5}$ and
					$\Set*{\vec v\in\R^2 \given
					\vec v=k\mat{1\\-1}\text{ for some }k\in\R}$
				\item $\mat{-1 \\ 2 \\ 1}$ and
					$\Set*{\vec v\in\R^3\given
						\vec v=i\mat{-2 \\ 1 \\ 0} +j\mat{3 \\ 0 \\ 1}
						\text{ for some }i,j\in\R
					}$
					\item
						$\mat{3 \\ -1}$ and $\Set*{ \vec v\in\R^2\given
							\vec v=k\mat{-6 \\ 2}\text{ for some } k\in\R
						}$
					\item
						$\mat{5 \\ 4}$ and
						$\Set*{\vec v\in\R^2\given
							\vec v=j\mat{5 \\ -4}\text{ for some }j\in\R
						}$
					\item $\mat{2 \\ 1 \\ -1}$ and
						$\Set*{\vec v\in\R^3
							\given \vec v= \mat{0 \\ 3 \\ -7} +r\mat{1 \\ -1 \\ 3}
							\text{ for some }r\in\R
						}$
					\item $\mat{1 \\ 0 \\ 1}$ and
						$\Set*{\vec v\in\R^3\given
							\vec v= j\mat{2 \\ 0 \\ 1} +k\mat{-3 \\ -1 \\ 1}
							\text{ for some }j,k\in\R
						}$
			\end{enumerate}

		% Q3 solution
		\begin{solution}
		    \begin{enumerate}
		        \item Yes, take $k=5$.
		        \item Yes, take $i=2,j=1$.
		        \item Yes, take $k=-\frac{1}{2}$.
		        \item No.
		        \item Yes, take $r=2$.
		        \item No.
		    \end{enumerate}
		\end{solution}

		% Conceptual (3 questions)
		\prob
			Let $X$ be the set of all positive even integers. Let $Y$ be the set of
			all positive multiples of $3$.
			\begin{enumerate}
				\item Write $X$ and $Y$ in set-builder notation.
				\item Determine what $X\cap Y$ is. Write your solution in two ways: in
					plain English words and also in set-builder notation.
			\end{enumerate}
		\prob
			% Purpose: get students to understand (the beginnings of) scale-invariance
			% of bases, as well as carefully reading mathematical expressions.
			Draw the following subsets of $\R^2$ and then determine which are equal or
			subsets of each other.
			\begin{enumerate}
				\item $A=\Set*{\vec v\in\R^2\given \vec v=n\mat{2\\1} \text{ for some }n\in\Z}$
				\item $B=\Set*{\vec v\in\R^2\given \vec v=t\mat{4\\2}\text{ for some }t\in\R}$
				\item $C=\Set*{\vec v\in\R^2\given \vec v=n\mat{4\\2} \text{ for some }n\in\Z}$
				\item $D=\Set*{\vec v\in\R^2\given \vec v=t\mat{2\\1}\text{ for some }t\in\R}$
			\end{enumerate}
			\begin{solution}
				We have $C\subseteq A$, $A\subseteq B$, and $B=D$.
			\end{solution}
		\prob
			Suppose the vector $\vec v$ can be written as a linear combination of the
			vectors $\vec a$, $\vec b$, and $\vec c$. Can $\vec v$ be written as a
			linear combination of the vectors $2\vec a$, $-\vec b$ and
			$\frac12\vec c$? Why or why not?
		\prob
			Demonstrate with a picture why $\vec v + \vec w = \vec w + \vec v$.
			\begin{solution}
				You get a parallelogram with side lengths $\norm{\vec v}$ and
				$\norm{\vec w}$. This is why this property is sometimes called the
				\emph{parallelogram law}.
			\end{solution}
		\prob Let $\vec a=\mat{1\\2}$, $\vec b=\mat{2\\4}$, $\vec c=\xhat+3\yhat$, and $\vec d=\vec a+\vec c$.
		\begin{enumerate}
			\item Is $\xhat$ a linear combination of $\vec a$ and $\vec b$?
			\item Is $\vec d$ a linear combination of $\vec a$ and $\vec b$?
			\item Is $\vec p=\mat{1\\1}$ a linear combination of $\vec a$ and $\vec c$?
			\item Is $\vec q=\mat{-3\\3}$ a linear combination of $\vec a$, $\vec b$, $\vec c$, and $\vec d$?
		\end{enumerate}
		
		% Q5 Solutions
		\begin{solution}
            \begin{enumerate}
    		    \item No.
    		    \item No.
    		    \item Yes.
    		    \item Yes.
		    \end{enumerate}
		\end{solution}
		% Challenge (3 questions)
		\prob
			Prove that for any sets $X$, $Y$ and $Z$ that:
			\begin{enumerate}
				\item $X \cap (Y \cup Z) = (X\cap Y) \cup (X\cap Z)$
				\item $X \cup (Y \cap Z) = (X\cup Y) \cap (X\cup Z)$
			\end{enumerate}
	\end{problist}
\end{exercises}
