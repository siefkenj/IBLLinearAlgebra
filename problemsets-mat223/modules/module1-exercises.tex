\begin{exercises}
		% Topics:
		% Sets, set builder notation, set operations,
		% vectors \& scalars, vector notation, vectors \& points, vector arithmetic,
		% coordinates \& the standard basis, higher dimensions,
	\begin{problist}
		\ProvideDocumentCommand{\vv}{}{\vec{v}}
		% Computation (4 questions)
		\prob[\beezer[VO.C10]] 
		\begin{enumerate}
			\item
				Write the following as a column matrix:
				\begin{enumerate}
					\item $4\sbasis{1} -3\sbasis{3} +2\sbasis{2} -2\sbasis{1}\in\R^3$.
					\item $1\sbasis{2} +\sbasis{1} -5\sbasis{2} \in\R^2$.
				\end{enumerate}
			\item
				Write the following as a linear combination of the standard basis
				vectors $\sbasis{1}, \sbasis{2}, \sbasis{3},\dots$:
				\begin{enumerate}
					\item $\mat{0\\-3\\2}$
					\item $\displaystyle
						\mat{-2\\5\\4} + (-1)\mat{-1\\2\\ 5} + \frac12\mat{2\\0\\2}$
					\item $\displaystyle
						2\mat{-1\\-2\\0\\ 5}
						+1\mat{3\\4\\-2\\ 1}
						+\left(-\frac13\right)\mat{0\\-3\\0\\ 6}$
				\end{enumerate}
		\end{enumerate}
		\prob[\hefferon[2.21,2.22]]
			Decide if the vector is in the set. If it is, what value of the
			parameters produce that vector?
			\begin{enumerate}
				\item
					$\mat{5\\-5}$ and
					$\Set*{\vv\in\R^2 \given
					\vv=k\mat{1\\-1}\text{ for some }k\in\R}$
				\item $\mat{-1 \\ 2 \\ 1}$ and
					$\Set*{\vv\in\R^3\given
						\vv=i\mat{-2 \\ 1 \\ 0} +j\mat{3 \\ 0 \\ 1}
						\text{ for some }i,j\in\R
					}$
					\item
						$\mat{3 \\ -1}$ and $\Set*{ \vv\in\R^2\given
							\vv=k\mat{-6 \\ 2}\text{ for some } k\in\R
						}$
					\item
						$\mat{5 \\ 4}$ and
						$\Set*{\vv\in\R^2\given
							\vv=j\mat{5 \\ -4}\text{ for some }j\in\R
						}$
					\item $\mat{2 \\ 1 \\ -1}$ and
						$\Set*{\vv\in\R^3
							\given \vv= \mat{0 \\ 3 \\ -7} +r\mat{1 \\ -1 \\ 3}
							\text{ for some }r\in\R
						}$
					\item $\mat{1 \\ 0 \\ 1}$ and
						$\Set*{\vv\in\R^3\given
							\vv= j\mat{2 \\ 0 \\ 1} +k\mat{-3 \\ -1 \\ 1}
							\text{ for some }j,k\in\R
						}$
			\end{enumerate}
		% Conceptual (3 questions)
		\prob
			Let $X$ be the set of all positive even integers. Let $Y$ be the set of
			all positive multiples of $3$.
			\begin{enumerate}
				\item Write $X$ and $Y$ in set-builder notation.
				\item Determine what $X\cap Y$ is. Write your solution in two ways: in
					plain English words and also in set-builder notation.
			\end{enumerate}
		\prob
			% Purpose: get students to understand (the beginnings of) scale-invariance
			% of bases, as well as carefully reading mathematical expressions.
			Draw the following subsets of $\R^2$:
			\begin{enumerate}
				\item $A=\Set*{\vv\in\R^2\given \vv=n\mat{2\\1}
					\text{ for some }n\in\Z}$
				\item $B=\Set*{\vv\in\R^2\given \vv=t\mat{4\\2}\text{ for some }t\in\R}$
				\item $C=\Set*{\vv\in\R^2\given \vv=n\mat{4\\2}
					\text{ for some }n\in\Z}$
				\item $D=\Set*{\vv\in\R^2\given \vv=t\mat{2\\1}\text{ for some }t\in\R}$
			\end{enumerate}
			For each pair of sets from $A$, $B$, $C$, and $D$, determine whether one
			is a subset of the other, if they are equal, or neither.
			\begin{solution}
				We have $C\subseteq A$, $A\subseteq B$, and $B=D$.
			\end{solution}
		\prob
			Suppose vector $\vec v$ can be written as a linear combination of the
			vectors $\vec a$, $\vec b$, and $\vec c$. Can $\vec v$ be written as a
			linear combination of the vectors $2\vec a$, $-\vec b$ and
			$\frac12\vec c$? Why or why not?
		\prob
			Demonstrate with a picture why $\vec v + \vec w = \vec w + \vec v$.
			\begin{solution}
				You get a parallelogram with side lengths $\norm{\vec v}$ and
				$\norm{\vec w}$. This is why this property is sometimes called the
				\emph{parallelogram law}.
			\end{solution}
		% Challenge (3 questions)
		\prob
			Prove that for any sets $X$, $Y$ and $Z$ that:
			\begin{enumerate}
				\item $X \cap (Y \cup Z) = (X\cap Y) \cup (X\cap Z)$
				\item $X \cup (Y \cap Z) = (X\cup Y) \cap (X\cup Z)$
			\end{enumerate}
	\end{problist}
\end{exercises}
