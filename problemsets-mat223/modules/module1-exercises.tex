\begin{exercises}
		% Topics:
		% Sets, set builder notation, set operations,
		% vectors \& scalars, vector notation, vectors \& points, vector arithmetic,
		% coordinates \& the standard basis, higher dimensions,
	\begin{problist}
		% Computation (4 questions)
		\prob
		\begin{enumerate}
			\item
				Write the following vectors as column vectors.
				\begin{enumerate}
					\item $4\xhat -3\zhat +2\yhat -2\xhat\in\R^3$.
					\item $\yhat +\xhat -5\yhat \in\R^2$.
				\end{enumerate}
			\item
				Write the following vectors as linear combinations of $\xhat$, $\yhat$, and
				$\zhat$.
				\begin{enumerate}
					\item $\mat{1\\-2\\3}$.
					\item $\mat{-2\\5\\4} + \mat{1\\-2\\ -5} + \mat{1\\0\\1}$.
				\end{enumerate}
		\end{enumerate}
		\prob
		Compute
		\[
			3\mat{2\\-1\\1\\1\\0}+
			(-2)\mat{1\\2\\-7\\3\\0}+
			\mat{-3\\3\\9\\2\\2}
		\]
		\prob[\hefferon[2.21,2.22]]
			Decide if the vector is in the set. If it is, what value of the
			parameters produce that vector?
			\begin{enumerate}
				\item
					$\mat{5\\-5}$ and
					$\Set*{\vec v\in\R^2 \given
					\vec v=k\mat{1\\-1}\text{ for some }k\in\R}$
				\item $\mat{-1 \\ 2 \\ 1}$ and
					$\Set*{\vec v\in\R^3\given
						\vec v=i\mat{-2 \\ 1 \\ 0} +j\mat{3 \\ 0 \\ 1}
						\text{ for some }i,j\in\R
					}$
					\item
						$\mat{3 \\ -1}$ and $\Set*{ \vec v\in\R^2\given
							\vec v=k\mat{-6 \\ 2}\text{ for some } k\in\R
						}$
					\item
						$\mat{5 \\ 4}$ and
						$\Set*{\vec v\in\R^2\given
							\vec v=j\mat{5 \\ -4}\text{ for some }j\in\R
						}$
					\item $\mat{2 \\ 1 \\ -1}$ and
						$\Set*{\vec v\in\R^3
							\given \vec v= \mat{0 \\ 3 \\ -7} +r\mat{1 \\ -1 \\ 3}
							\text{ for some }r\in\R
						}$
					\item $\mat{1 \\ 0 \\ 1}$ and
						$\Set*{\vec v\in\R^3\given
							\vec v= j\mat{2 \\ 0 \\ 1} +k\mat{-3 \\ -1 \\ 1}
							\text{ for some }j,k\in\R
						}$
			\end{enumerate}
		% Conceptual (3 questions)
		\prob
			% Purpose: get students to understand (the beginnings of) scale-invariance
			% of bases, as well as carefully reading mathematical expressions.
			Draw the following sets of $\R^2$ and then determine which are equal or subsets of each other.
			\begin{enumerate}
				\item $A=\Set*{\vec v\in\R^2\given \vec v=n\mat{2\\1}\text{ for some integer }n\in\Z}$
				\item $B=\Set*{\vec v\in\R^2\given \vec v=t\mat{4\\2}\text{ for some }t\in\R}$
				\item $C=\Set*{\vec v\in\R^2\given \vec v=n\mat{4\\2}\text{ for some integer }n\in\Z}$
				\item $D=\Set*{\vec v\in\R^2\given \vec v=t\mat{2\\1}\text{ for some }t\in\R}$
			\end{enumerate}
			\begin{solution}
				We have $C\subseteq A$, $A\subseteq B$, and $B=D$.
			\end{solution}
		\prob Let $\vec a=\mat{1\\2}$, $\vec b=\mat{2\\4}$, $\vec c=\xhat+3\yhat$, and $\vec d=\vec a+\vec c$.
		\begin{enumerate}
			\item Is $\xhat$ a linear combination of $\vec a$ and $\vec b$?
			\item Is $\vec d$ a linear combination of $\vec a$ and $\vec b$?
			\item Is $\vec p=\mat{1\\1}$ a linear combination of $\vec a$ and $\vec c$?
			\item Is $\vec q=\mat{-3\\3}$ a linear combination of $\vec a$, $\vec b$, $\vec c$, and $\vec d$?
		\end{enumerate}
		% Challenge (3 questions)
	\end{problist}
\end{exercises}
