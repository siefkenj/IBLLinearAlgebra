\begin{exercises}
	\begin{problist}
		\prob Let $\mathcal T:\R^{2}\to\R^{2}$ be defined by
		$\mathcal T\mat{x\\y}=\matc{3x-y\\x-\tfrac{1}{4}y}$. Find the volume
		of $\mathcal T(C_{2})$.

		\prob Let $\mathcal S:\R^{3}\to\R^{3}$ be defined by
		$\mathcal S\mat{x\\y\\z}=\matc{2x+y+z\\x-\tfrac{1}{2}y\\z}$. Find
		the volume of $\mathcal S(C_{3})$.

		\prob Let $\mathcal T:\R^{2}\to\R^{2}$ be defined by
		$\mathcal T\mat{x\\y}=\matc{x+2y\\-x-y}$.
		\begin{enumerate}
			\item draw $\mathcal{E}$ and $\mathcal{T}(\mathcal{E})$ and
				then determine whether $\mathcal{T}$ is orientation
				preserving or orientation reversing.

			\item Find $\det(\mathcal T)$.
		\end{enumerate}

		\prob For each linear transformation defined below, find its
		determinant.
		\begin{enumerate}
			\item $\mathcal S:\R^{2}\to\R^{2}$, where $\mathcal S$
				shortens
				every vector by a factor of $\tfrac{2}{3}$.

			\item $\mathcal R:\R^{2}\to\R^{2}$, where $\mathcal R$
				is rotation counter-clockwise by $90^{\circ}$.

			\item $\mathcal F:\R^{2}\to\R^{2}$, where $\mathcal F$
				is reflection across the line $y=-x$.

			\item $\mathcal G:\R^{2}\to\R^{2}$, where
				$\mathcal G(\vec x)=\mathcal{P}(\vec x)+
				\mathcal{Q}(\vec x)$ and where $\mathcal{P}$ is projection onto the
				line $y=x$ and $\mathcal{Q}$ is projection onto
				the line $y=-\tfrac{1}{2}x$.

			\item $\mathcal T:\R^{3}\to\R^{3}$, where
				$\mathcal T\mat{x\\y\\z}=\matc{x-y+z\\z+x-\tfrac{1}{3}y\\z}$.

			\item $\mathcal J:\R^{3}\to\R^{3}$, where
				$\mathcal J\mat{x\\y\\z}=\matc{0\\0\\x+y+z}$.

			\item $\mathcal K\circ \mathcal H:R^{2}\to\R^{2}$, where
				$\mathcal H\mat{x\\y}=\matc{x+2y\\-x-y}$,

				and $\mathcal K\mat{x\\y}=\matc{-x-2y\\x+y}$.
		\end{enumerate}

		\prob Let $A=\mat{2&3\\1&5}$.
		\begin{enumerate}
			\item Use elementary matrices to find $\det(A)$.

			\item Draw a picture of the parallelogram given by the rows
				of $A$. 
				\label{PROBMOD14-rows}

			\item Draw a picture of the parallelogram given by the columns
				of $A$.
				\label{PROBMOD14-cols}
			\item How do the areas of the parallelograms drawn in parts \ref{PROBMOD14-rows} and
				\ref{PROBMOD14-cols} relate?
		\end{enumerate}

		\prob Let $A=\mat{1&2&0\\0&2&1\\1&2&3}$.
		\begin{enumerate}
			\item \label{Module14-q8} Use elementary matrices to
				find $\det(A)$.

			\item Find $\det(A^{-1})$.

			\item Find $\det(A^{T})$, and compare your answer with
				\ref{Module14-q8}. Are they the same? Explain.
		\end{enumerate}

		\prob Let $A$ be an $n \times n$ matrix that can be decomposed into
		the product of elementary matrices.
		\begin{enumerate}
			\item What is $\Rank(A)$? Justify your answer.

			\item What is $\Null(A^{-1})$? Justify your answer.
		\end{enumerate}
	\end{problist}
\end{exercises}

