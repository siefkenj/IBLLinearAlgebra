\documentclass{problemset}
\usepackage{amsmath}

\usepackage{lipsum}
%\usepackage{showframe}
\usepackage{layout}


\usepackage[charter,cal=cmcal]{mathdesign} %different font
\usepackage{microtype}
\usepackage{amsmath}
%\usepackage{amsfonts}
%\usepackage{amssymb}
\usepackage{graphicx}
\usepackage[inline]{enumitem}
\usepackage{xparse}
\usepackage{ifthen}
\usepackage{graphicx}
\usepackage{caption}
\usepackage{subcaption}
\usepackage{color}
\usepackage{tikz}
\usepackage{fancyhdr}
\usepackage{calc}
\usepackage[hidelinks]{hyperref}

\usepackage{pgfplots}
\pgfplotsset{compat=newest}
%%%
% Useful Linear Algebra macros
%%%
\newcommand{\ul}{$\underline{\phantom{xxx}}$}
\newcommand{\ull}{\underline{\phantom{xxx}}}
\newcommand{\xh}{{\hat {\mathbf x}}}
\newcommand{\yh}{{\hat {\mathbf y}}}
\newcommand{\zh}{{\hat {\mathbf z}}}
\newcommand{\R}{\mathbb{R}}
\newcommand{\Z}{\mathbb{Z}}
\newcommand{\N}{\mathbb{N}}
\newcommand{\proj}{\mathrm{proj}}
\newcommand{\Proj}{\mathrm{proj}}
\newcommand{\Perp}{\mathrm{perp}}
\renewcommand{\span}{\mathrm{span}\,}
\newcommand{\Span}{\mathrm{span}\,}
\newcommand{\Img}{\mathrm{img}\,}
\newcommand{\Null}{\mathrm{null}\,}
\newcommand{\Range}{\mathrm{range}\,}
\newcommand{\rref}{\mathrm{rref}}
\newcommand{\rank}{\mathrm{rank}}
\newcommand{\Rank}{\mathrm{rank}}
\newcommand{\nnul}{\mathrm{nullity}}
\newcommand{\mat}[1]{\begin{bmatrix}#1\end{bmatrix}}
\newcommand{\chr}{\mathrm{char}}
\renewcommand{\d}{\mathrm{d}}

%\tcbuselibrary{skins}
%\usetikzlibrary{shadings}


%%%
% Set up the margins to use a fairly large area of the page
%%%
\textwidth=6in
\topmargin=-1in
\textheight=10in
\parskip=.07in
\parindent=0in

\begin{document}
\pagestyle{empty}

\begin{center}
{\huge\bf Inquiry Based Vector Calculus}\\

\vspace{.7in}
{
\it \copyright\,Jason Siefken, 2016 \\
Creative Commons By-Attribution Share-Alike\, \makebox(30,5){\includegraphics[height=1.2em]{by-sa.pdf}}
}
\end{center}

\section*{About the Document}

	This document was originally designed in the spring of 2016 to guide students
	through an ten week Linear Algebra course (Math 281-3) at
	Northwestern University.  

	A typical class day using the problem-sets:
	\begin{enumerate}
		\item {\bf Introduction by instructor.} This may involve giving a definition,
			a broader context for the day's topics, or answering questions.
		\item {\bf Students work on problems.} Students work individually or in pairs
			on the prescribed problem.  During this time the instructor moves around
			the room addressing questions that students may have and giving one-on-one
			coaching.
		\item {\bf Instructor intervention.} If most students have successfully solved the 
			problem, the instructor regroups the class by providing a concise 
			explanation so that everyone is ready to move to the next concept.  This
			is also time for the instructor to ensure that everyone has understood the
			main point of the exercise (since it is sometimes easy to do some computation
			while being oblivious to the larger context).

			If students are having trouble, the instructor can give hints to the group,
			and additional guidance to ensure the students don't get frustrated
			to the point of giving up.
		\item {\bf Repeat step 2.}
	\end{enumerate}

	Using this format, students are working (and happily so) most of the class.
	Further, they are especially primed to hear the insights of the instructor, 
	having already invested substantially into each problem.

	This problem-set is geared towards concepts instead of computation, though some problems
	focus on simple computation.

	{\bf License}  This document is licensed under the Creative Commons
	By-Attribution Share-Alike License.  That means, you are free to use,
	copy, and modify this document provided that you provide attribution
	to the previous copyright holders and you release your derivative work 
	under the same license.  Full text of the license is at \url{http://creativecommons.org/licenses/by-sa/4.0/}

	If you modify this document, you may add your name to the copyright list.  Also,
	if you think your contributions would be helpful to others, consider making a pull
	requestion, or opening an \emph{issue} at 
	\url{https://github.com/siefkenj/IBLLinearAlegbra}


\newpage

\setcounter{page}{1}
\pagestyle{fancy}
\rfoot{\footnotesize\it \copyright\,Jason Siefken, 2015 \ \makebox(30,5){\includegraphics[height=1.2em]{by-sa.pdf}}}
\renewcommand{\headrulewidth}{0pt}

\section*{Sets of Vectors}
	\vspace{-2em}
	\question
		Write the following sets in set-builder notation
	\begin{parts}
			\item The subset $A\subseteq \R$ of real numbers larger than $\sqrt{2}$.
			\item The subset $B\subseteq \R^2$ of vectors whose first coordinate
			is twice the second.
	\end{parts}

	\begin{definition}[Unions \& Intersections]
		Two common set operations are \emph{unions} and \emph{intersections}.  
		Let $X$ and $Y$ be sets.

		\hfill\begin{minipage}{\dimexpr\textwidth-3cm}
		\begin{itemize}
			\item[(union)] $X\cup Y = \{a:a\in X\text{ or }a\in Y\}$.
			\item[(intersection)] $X\cap Y = \{a: a\in X\text{ and }a\in Y\}$.
		\end{itemize}
		\end{minipage}
	\end{definition}

	\question
	Let $X=\{1,2,3\}$ and $Y=\{2,3,4,5\}$ and $Z=\{4,5,6\}$.  Compute
	\begin{parts}
		\item $X\cup Y$
		\item $X\cap Y$
		\item $X\cup Y\cup Z$
		\item $X\cap Y\cap Z$
	\end{parts}

	\question
	Draw the following subsets of $\R^2$.
	\begin{parts}
		\item $V=\left\{\vec x\in\R^2:\vec x=\begin{bmatrix}0\\t\end{bmatrix}\text{ for some }t\in\R\right\}$.
		\item $H=\left\{\vec x\in\R^2:\vec x=\begin{bmatrix}t\\0\end{bmatrix}\text{ for some }t\in\R\right\}$.
		\item $J=\left\{\vec x\in\R^2:\vec x=t\begin{bmatrix}1\\1\end{bmatrix}\text{ for some }t\in\R\right\}$.
		\item $V\cup H$.
		\item $V\cap H$.
		\item Does $V\cup H=\R^2$?
	\end{parts}


	\vspace{-1em}
\section*{Linear Combinations, Span, and Linear Independence}
	\vspace{-1em}

	\begin{definition}[Linear Combination]
		A \emph{linear combination} of the vectors $\vec v_1,\vec v_2,\ldots,\vec v_n$ is
		a vector
		\[
			\vec w = \alpha_1\vec v_1+\alpha_2\vec v_2+\cdots+\alpha_n\vec v_n
		\]
		where $\alpha_1,\alpha_2,\ldots,\alpha_n$ are scalars.
	\end{definition}

	\question
	Let $\vec v_1=\begin{bmatrix}1\\1\end{bmatrix}$, $\vec v_2=\begin{bmatrix}1\\-1\end{bmatrix}$, and $\vec w=2\vec v_1+\vec v_2$.
	\begin{parts}
		\item Write the coordinates of $\vec w$.
		\item Draw a picture with $\vec w$, $\vec v_1$, and $\vec v_2$.
		\item Is $\mat{3\\3}$ a linear combination of $\vec v_1$ and $\vec v_2$?
		\item Is $\mat{0\\0}$ a linear combination of $\vec v_1$ and $\vec v_2$?
		\item Is $\mat{4\\0}$ a linear combination of $\vec v_1$ and $\vec v_2$?
		\item Can you find a vector in $\R^2$ that isn't a linear combination of
		$\vec v_1$ and $\vec v_2$?
		\item Can you find a vector in $\R^2$ that isn't a linear combination of
		$\vec v_1$?
	\end{parts}
	
	\begin{definition}[Span]
		The \emph{span} of a set of vectors $V$ is the set of
		all linear combinations of vectors in $V$.  That is,
		\[
			\Span V = \{\vec v:\vec v=\alpha_1\vec v_1+\alpha_2\vec v_2 + \cdots 
			+\alpha_n\vec v_n\text{ for some }\vec v_1,\vec v_2,\ldots,\vec v_n\in V
			\text{ and scalars }\alpha_1,\alpha_2,\ldots,\alpha_n\}.
		\]
	\end{definition}

	\question
	Let $\vec v_1=\mat{1\\1}$, $\vec v_2=\mat{1\\-1}$, and $\vec v_3=\mat{2\\2}$.
	\begin{parts}
		\item Draw $\Span\{\vec v_1\}$.
		\item Draw $\Span\{\vec v_2\}$.
		\item Describe $\Span\{\vec v_1,\vec v_2\}$.
		\item Describe $\Span\{\vec v_1,\vec v_3\}$.
		\item Describe $\Span\{\vec v_1,\vec v_2,\vec v_3\}$.
	\end{parts}

	\question
	Give an example of:
	\begin{parts}
		\item two vectors in $\R^3$ that span a plane;
		\item two vectors in $\R^3$ that span a line;
		\item four vectors in $\R^3$ that span a plane;
		\item a set of 50 vectors in $\R^3$ whose span is the line
		through the origin and the point $\mat{1\\2\\-3}$. \\
	\end{parts}

	In some sets, every vector is essential for computing a span.  In others,
	there are ``excess'' vectors.  This leads us to the concept of 
	linear independence.

	\begin{definition}
		We say $\{\vec v_1,\vec v_2,\ldots,\vec v_n\}$ is
		\emph{linearly dependent} if for at least one $i$,
		\[
			\vec v_i\in\span\{\vec v_1,\vec v_2,\ldots,\vec v_{i-1},
			\vec v_{i+1},\ldots,\vec v_n\},
		\]
		and a set is \emph{linearly independent} otherwise.
	\end{definition}

	\question
		Let $\vec u=\mat{1\\0\\0}$, $\vec v=\mat{0\\1\\0}$, and $\vec w=\mat{1\\1\\0}$.
	\begin{parts}
		\item Describe $\Span\{\vec u,\vec v,\vec w\}$.
		\item Is $\{\vec u,\vec v,\vec w\}$ linearly independent?  Why or why not?
	\end{parts}

	Let $X=\{\vec u,\vec v,\vec w\}$.

	\begin{parts}[resume]
		\item Give a subset $Y\subseteq X$ so that $\Span Y=\Span X$ and $Y$ is
		linearly independent.
		\item Give a subset $Z\subseteq X$ so that $\Span Z=\Span X$ and $Z$ is
		linearly independent and $Z\neq Y$.
	\end{parts}
	
	\begin{definition}[Trivial Linear Combination]
	We say a linear combination 
	$a_1\vec v_1+a_2\vec v_2+\cdots +a_n\vec v_n$
	is \emph{trivial} if $a_1=a_2=\cdots=a_n=0$.
	\end{definition}
	
	\question
		Recall $\vec u=\mat{1\\0\\0}$, $\vec v=\mat{0\\1\\0}$, and $\vec w=\mat{1\\1\\0}$.
	\begin{parts}
		\item Consider the linearly dependent 
		set $\{\vec u,\vec v,\vec w\}$ (where $\vec u,\vec v,\vec w$
		are defined as above).  Can you write $\vec 0$
		as a non-trivial linear combination of vectors in this set?
		\item Consider the linearly independent 
		set $\{\vec u,\vec v\}$.  Can you write $\vec 0$
		as a non-trivial linear combination of vectors in this set?
	\end{parts}

	We now have an equivalent definition of linear dependence.

	\begin{definition}
	$\{\vec v_1,\vec v_2,\ldots,\vec v_n\}$ is
	\emph{linearly dependent} if there is a non-trivial
	linear combination of $\vec v_1,\ldots,\vec v_n$ that
	equals the zero vector.
	\end{definition}

	\question
	\begin{parts}
		\item Explain how this new definition implies the old one.
		\item Explain how the old definition implies this new one.
	\end{parts}

	Since have old def $\implies$ new def, and new def $\implies$ old def ($\implies$
	should be read aloud as `implies'), the two definitions
	are \emph{equivalent} (which we write as new def $\iff$ old def).


	\question
	Suppose for some unknown $\vec u, \vec v, \vec w$, and $\vec a$,
	\[
		\vec a = 3\vec u+2\vec v +\vec w\qquad \text{and}\qquad 
		\vec a = 2\vec u+\vec v -\vec w.
	\]
	\begin{parts}
		\item Could the set $\{\vec u,\vec v,\vec w\}$ be linearly
		independent?
	\end{parts}
	Suppose that
	\[
		\vec a = \vec u+6\vec r-\vec s
	\]
	is the \emph{only} way to write $\vec a$ using $\vec u,\vec r,\vec s$.
	\begin{parts}[resume]
		\item Is $\{\vec u,\vec r,\vec s\}$ linearly independent?
		\item Is $\{\vec u,\vec r\}$ linearly independent?
		\item Is $\{\vec u,\vec v,\vec w,\vec r\}$ linearly independent?
	\end{parts}

\section*{Subspaces and Bases}
	\begin{definition}
		A \emph{subspace} $V\subseteq \R^n$ is a subset such that
		\begin{enumerate}
			\item[(i)] $\vec u,\vec v\in V$ implies $\vec u+\vec v\in V$.
			\item[(ii)] $\vec u\in V$ implies $k\vec u\in V$ for all scalars $k$.
		\end{enumerate}
	\end{definition}

	Subspaces give a mathematically precise definition of a ``flat space through the origin.''

	\question
	For each set, draw it and explain whether or not it is a subspace of $\R^2$.
	\begin{parts}
		\item $A=\{\vec x\in\R^2:\vec x=\mat{a\\0}\text{ for some }a\in\Z\}$.
		\item $B=\{\vec x\in\R^2:\vec x\neq \mat{0\\0}\}$.
		\item $C=\{\vec x\in\R^2:\vec x=\mat{0\\t}\text{ for some }t\in\R\}$.
		\item $D=\{\vec x\in\R^2:\vec x=\mat{0\\t}+\mat{1\\1}\text{ for some }t\in\R\}$.
		\item $E=\{\vec x\in\R^2:\vec x=\mat{0\\t}\text{ or }\vec x=\mat{t\\0}\text{ for some }t\in\R\}$.
		\item $F=\{\vec x\in\R^2:\vec x=t\mat{3\\1}\text{ for some }t\in\R\}$.
		\item $G=\span\left\{\mat{1\\1}\right\}$.
		\item $H=\span\{\vec u,\vec v\}$ for some unknown vectors $\vec u,\vec v\in\R^2$.
	\end{parts}

	\begin{definition}
		A \emph{basis} for a subspace $V$ is a linearly independent set of vectors, $\mathcal B$,
		so that $\Span\mathcal B=V$.
	\end{definition}

	\question
	Let $\vec u=\mat{1\\0\\0}$, $\vec v=\mat{0\\1\\0}$, $\vec w=\mat{1\\1\\0}$, and $V=\span\{\vec u,\vec v,\vec w\}$.
	\begin{parts}
		\item Describe $V$.
		\item Is $\{\vec u,\vec v,\vec w\}$ a basis for $V$?  Why or why not?
		\item Give a basis for $V$.
		\item Give another basis for $V$.
		\item Is $\Span\{\vec u,\vec v\}$ a basis for $V$?  Why or why not?
	\end{parts}

	\begin{definition}
		The \emph{dimension} of a subspace $V$ is the number of elements in a basis for $V$.
	\end{definition}

	\begin{parts}[resume]
		\item What is the dimension of $V$?
	\end{parts}


	\question
	Let $\vec a=\mat{1\\2\\3}$, $\vec b=\mat{4\\5\\6}$, $\vec c=\mat{7\\8\\8}$ and 
	let $P=\span\{\vec a,\vec b\}$ and $Q=\span\{\vec b,\vec c\}$.
	\begin{parts}
		\item Give a basis for and the dimension of $P$.
		\item Give a basis for and the dimension of $Q$.
		\item Is $P\cap Q$ a subspace? If so, give a basis for it and its dimension.
		\item Is $P\cup Q$ a subspace? If so, give a basis for it and its dimension.
	\end{parts}



\end{document}
